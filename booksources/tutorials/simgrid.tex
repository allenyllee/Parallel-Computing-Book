% -*- latex -*-
%%%%%%%%%%%%%%%%%%%%%%%%%%%%%%%%%%%%%%%%%%%%%%%%%%%%%%%%%%%%%%%%
%%%%%%%%%%%%%%%%%%%%%%%%%%%%%%%%%%%%%%%%%%%%%%%%%%%%%%%%%%%%%%%%
%%%%
%%%% This text file is part of the source of 
%%%% `Parallel Programming in MPI and OpenMP'
%%%% by Victor Eijkhout, copyright 2012-8
%%%%
%%%% simgrid.tex : about grid-in-a-box simulation
%%%%
%%%%%%%%%%%%%%%%%%%%%%%%%%%%%%%%%%%%%%%%%%%%%%%%%%%%%%%%%%%%%%%%
%%%%%%%%%%%%%%%%%%%%%%%%%%%%%%%%%%%%%%%%%%%%%%%%%%%%%%%%%%%%%%%%

\index{SimGrid|(}

Many readers of this book will have access to some sort of parallel
machine so that they can run simulations, maybe even some realistic
scaling studies. However, not many people will have access to more
than one cluster type so that they can evaluate the influence of the
\indexterm{interconnect}. Even then, for didactic purposes one would
often wish for interconnect types (fully connected, linear processor
array) that are unlikely to be available.

In order to explore architectural issues pertaining to the network, we
then resort to a simulation tool, \emph{SimGrid}.

\heading{Installation}

\heading{Compilation}

You write plain MPI files, but compile them with the
\indextermbus{SimGrid}{compiler} \indextermtt{smpicc}.

\heading{Running}

SimGrid has its own version of \n{mpirun}: \indextermtt{smpirun}. You
need to supply this with options:
\begin{itemize}
\item \n{-np 123456} for the number of (virtual) processors;
\item \n{-hostfile simgridhostfile} which lists the names of these
  processors. You can basically make these up, but are defined in:
\item \n{-platform arch.xml} which defines the connectivity between
  the processors.
\end{itemize}
For instance, with a hostfile of 8 hosts, a linearly connected network
would be defined as:
\begin{verbatim}
<?xml version='1.0'?>
<!DOCTYPE platform SYSTEM "http://simgrid.gforge.inria.fr/simgrid/simgrid.dtd">

<platform version="4">

<zone id="first zone" routing="Floyd">
  <!-- the resources -->
  <host id="host1" speed="1Mf"/>
  <host id="host2" speed="1Mf"/>
  <host id="host3" speed="1Mf"/>
  <host id="host4" speed="1Mf"/>
  <host id="host5" speed="1Mf"/>
  <host id="host6" speed="1Mf"/>
  <host id="host7" speed="1Mf"/>
  <host id="host8" speed="1Mf"/>
  <link id="link1" bandwidth="125MBps" latency="100us"/>
  <!-- the routing: specify how the hosts are interconnected -->
  <route src="host1" dst="host2"><link_ctn id="link1"/></route>
  <route src="host2" dst="host3"><link_ctn id="link1"/></route>
  <route src="host3" dst="host4"><link_ctn id="link1"/></route>
  <route src="host4" dst="host5"><link_ctn id="link1"/></route>
  <route src="host5" dst="host6"><link_ctn id="link1"/></route>
  <route src="host6" dst="host7"><link_ctn id="link1"/></route>
  <route src="host7" dst="host8"><link_ctn id="link1"/></route>
</zone>

</platform>
\end{verbatim}
(such files are easily generated with a shell script).

The \n{Floyd} designation of the routing means that any route using
the transitive closure of the paths given can be used.
It is also possible to use \n{routing="Full"} which requires full
specification of all pairs that can communicate.


\index{SimGrid|)}
