% -*- latex -*-
%%%%%%%%%%%%%%%%%%%%%%%%%%%%%%%%%%%%%%%%%%%%%%%%%%%%%%%%%%%%%%%%
%%%%%%%%%%%%%%%%%%%%%%%%%%%%%%%%%%%%%%%%%%%%%%%%%%%%%%%%%%%%%%%%
%%%%
%%%% This text file is part of the source of 
%%%% `Parallel Programming in MPI and OpenMP'
%%%% by Victor Eijkhout, copyright 2012-6
%%%%
%%%% omp-sync.tex : synchronization
%%%%
%%%%%%%%%%%%%%%%%%%%%%%%%%%%%%%%%%%%%%%%%%%%%%%%%%%%%%%%%%%%%%%%
%%%%%%%%%%%%%%%%%%%%%%%%%%%%%%%%%%%%%%%%%%%%%%%%%%%%%%%%%%%%%%%%

\index{synchronization!in OpenMP|(}

In the constructs for declaring parallel regions above, you had little control over 
in what order threads executed the work they were assigned.
This section will discuss \emph{synchronization} constructs: ways of telling
threads to bring a certain order to the sequence in which they do things.

\begin{itemize}
\item \texttt{critical}: a section of code can only be executed by one
  thread at a time; see~\ref{sec:critical}.
\item \texttt{atomic} Update of a single memory location. Only certain
  specified syntax pattterns are supported. This was added in order to
  be able to use hardware support for atomic updates.
\item \texttt{barrier}: section~\ref{sec:ompbarrier}.
\item \texttt{ordered}: section~\ref{sec:omp-ordered}.
\item locks: section \ref{ompref:locks}.
\item \texttt{flush}: section \ref{sec:omp:flush}.
\item \texttt{nowait}: section~\ref{sec:omp-nowait}.
\end{itemize}

\Level 0 {Barrier}
\label{sec:ompbarrier}

A barrier defines a point in the code where all active threads will stop
until all threads have arrived at that point. With this, you can guarantee that
certain calculations are finished. For instance, in this code snippet, computation 
of~\n{y} can not proceed until another thread has computed its value of~\n{x}.
\begin{verbatim}
#pragma omp parallel 
{
  int mytid = omp_get_thread_num();
  x[mytid] = some_calculation();
  y[mytid] = x[mytid]+x[mytid+1];
}
\end{verbatim}
This can be guaranteed with a \indexpragma{barrier} pragma:
\begin{verbatim}
#pragma omp parallel 
{
  int mytid = omp_get_thread_num();
  x[mytid] = some_calculation();
#pragma omp barrier
  y[mytid] = x[mytid]+x[mytid+1];
}
\end{verbatim}

Apart from the barrier directive, which inserts an explicit barrier,
OpenMP has \emph{implicit barriers}\index{omp!barrier!implicit} after
a load sharing construct. Thus the following code is well defined:
\begin{verbatim}
#pragma omp parallel 
{
#pragma omp for
  for (int mytid=0; mytid<number_of_threads; mytid++)
    x[mytid] = some_calculation();
#pragma omp for
  for (int mytid=0; mytid<number_of_threads-1; mytid++)
    y[mytid] = x[mytid]+x[mytid+1];
}
\end{verbatim}

You can also put each parallel loop in a parallel region of its own,
but there is some overhead associated with creating and deleting the
team of threads in between the regions.

\Level 1 {Implicit barriers}

At the end of a parallel region the team of threads is dissolved and
only the master thread continues. Therefore, there is an
\emph{implicit barrier at the end of a parallel region}%
\index{parallel region!barrier at the end of}.

There is some \emph{barrier behaviour}\index{omp!for!barrier
  behaviour} associated with \n{omp for} loops and other
\emph{worksharing constructs}\index{worksharing constructs!implied
  barriers at} (see section~\ref{sec:workshare}).  For instance, there
is an \indexterm{implicit barrier} at the end of the loop. This
barrier behaviour can be cancelled with the \indexclause{nowait}
clause.

You will often see the idiom
\begin{verbatim}
#pragma omp parallel
{
#pragma omp for nowait
  for (i=0; i<N; i++)
    a[i] = // some expression
#pragma omp for
  for (i=0; i<N; i++)
    b[i] = ...... a[i] ......
\end{verbatim}
Here the \n{nowait} clause implies that threads can start on the second loop
while other threads are still working on the first. Since the two loops use the same
schedule here, an iteration that uses \n{a[i]} can indeed rely on it that that 
value has been computed.

\Level 0 {Mutual exclusion}

Sometimes it is necessary to let only one thread execute a piece of code.
Such a piece of code is called a \indexterm{critical section}, and
OpenMP has several mechanisms for realizing this.

The most common use of critical sections is to update a variable. Since updating
involves reading the old value, and writing back the new, this has the possibility
for a \indexterm{race condition}: another thread reads the current value
before the first can update it; the second thread the updates to the wrong value.

Critical sections are an easy way to turn an existing code into a correct parallel code.
However, there are disadvantages to this, and sometimes a more drastic rewrite
is called for.

\Level 1 {\texttt{critical} and \texttt{atomic}}
\label{sec:critical}

There are two pragmas for critical sections: \indexpragma{critical} and \indexpragma{atomic}.
The second one is more limited but has performance advantages.

The typical application of a critical section is to update a variable:
\begin{verbatim}
#pragma omp parallel
{
  int mytid = omp_get_thread_num();
  double tmp = some_function(mytid);
#pragma omp critical
  sum += tmp;
}
\end{verbatim}

\begin{exercise}
  Consider  a loop where each iteration updates a variable.
\begin{verbatim}
#pragma omp parallel for shared(result)
  for ( i ) {
      result += some_function_of(i);
  }
\end{verbatim}
  Discuss qualitatively
  the difference between:
  \begin{itemize}
  \item  turning the update statement into a critical section, versus
  \item letting the threads accumulate into a private variable \n{tmp} as above,
    and summing these after the loop.
  \end{itemize}  
  Do an Ahmdal-style quantitative analysis of the first case, assuming
  that you do $n$ iterations on $p$ threads, and each iteration has a
  critical section that takes a fraction~$f$.  Assume the number of
  iterations~$n$ is a multiple of the number of threads~$p$. Also
  assume the default static distribution of loop iterations over the
  threads.
\end{exercise}

A \n{critical} section works by acquiring a lock, which carries a substantial overhead.
Furthermore, if your code has multiple critical sections, they are all mutually exclusive:
if a thread is in one critical section, the other ones are all blocked.

On the other hand, the syntax for \n{atomic} sections is limited to the update
of a single memory location, but such sections
are not exclusive and they can be more efficient, since they assume that there is a hardware
mechanism for making them critical.

The problem with \n{critical} sections being mutually exclusive can be mitigated by naming them:
\begin{verbatim}
#pragma omp critical (optional_name_in_parens)
\end{verbatim}

\Level 0 {Locks}
\indexterm{lock|(textbf}

OpenMP also has the traditional mechanism of a \indexterm{lock}. A~lock is somewhat similar to 
a critical section: it guarantees that some instructions can only be performed by one
process at a time. However, a critical section is indeed about code; a~lock is about data.
With a lock you make sure that some data elements can only be touched by one process at a time.

One simple example of the use of locks is generation of a \indexterm{histogram}.
A~histogram consists of a number of bins, that get updated depending on some data.
Here is the basic structure of such a code:
\begin{verbatim}
int count[100];
float x = some_function();
int ix = (int)x;
if (ix>=100)
  error();
else
  count[ix]++;
\end{verbatim}
It would be possible to guard the last line:
\begin{verbatim}
#pragma omp critical
  count[ix]++;
\end{verbatim}
but that is unnecessarily restrictive. If there are enough bins in the
histogram, and if the \n{some_function} takes enough time, there are unlikely to be
conflicting writes. The solution then is to create an array of locks, with
one lock for each \n{count} location.

Create/destroy:
\begin{verbatim}
void omp_init_lock(omp_lock_t *lock);
void omp_destroy_lock(omp_lock_t *lock);
\end{verbatim}
Set and release:
\begin{verbatim}
void omp_set_lock(omp_lock_t *lock);
void omp_unset_lock(omp_lock_t *lock);
\end{verbatim}
Since the set call is blocking, there is also 
\begin{verbatim}
omp_test_lock();
\end{verbatim}

Unsetting a lock needs to be done by the thread that set it.

Lock operations implicitly have a \indexpragma{flush}.

\begin{exercise}
  \label{ex:loc-deadlock}
  %% https://computing.llnl.gov/tutorials/openMP/samples/C/omp_bug5.c
  %% I think this is fundamentally broken
  In the following code, one process sets array~A and then uses it to
  update~B; the other process sets array~B and then uses it to
  update~A.
  Argue that this code can deadlock. How could you fix this?
\begin{verbatim}
#pragma omp parallel shared(a, b, nthreads, locka, lockb)
  #pragma omp sections nowait
    {
    #pragma omp section
      {
      omp_set_lock(&locka);
      for (i=0; i<N; i++)
        a[i] = ..

      omp_set_lock(&lockb);
      for (i=0; i<N; i++)
        b[i] = .. a[i] ..
      omp_unset_lock(&lockb);
      omp_unset_lock(&locka);
      }

    #pragma omp section
      {
      omp_set_lock(&lockb);
      for (i=0; i<N; i++)
        b[i] = ...

      omp_set_lock(&locka);
      for (i=0; i<N; i++)
        a[i] = .. b[i] ..
      omp_unset_lock(&locka);
      omp_unset_lock(&lockb);
      }
    }  /* end of sections */
  }  /* end of parallel region */
\end{verbatim}
\end{exercise}

\Level 1 {Nested locks}

A lock as explained above can not be locked if it is already locked.
A~\indextermsub{nested}{lock} can be locked multiple times by the same
thread before being unlocked.

\begin{itemize}
\item \indextermttdef{omp_init_nest_lock}
\item \indextermttdef{omp_destroy_nest_lock}
\item \indextermttdef{omp_set_nest_lock}
\item \indextermttdef{omp_unset_nest_lock}
\item \indextermttdef{omp_test_nest_lock}
\end{itemize}

\indexterm{lock|)}

\Level 0 {Example: Fibonacci computation}
\index{Fibonacci sequence|(}

The \emph{Fibonacci sequence} is recursively defined as
\[ F(0)=1,\qquad F(1)=1,\qquad F(n)=F(n-1)+F(n-2)
\hbox{ for $n\geq2$}.
\]
We start by sketching the basic single-threaded solution.
The naive code looks like:
\begin{verbatim}
int main() {
  value = new int[nmax+1];
  value[0] = 1;
  value[1] = 1;
  fib(10);
}

int fib(int n) {
  int i, j, result;
  if (n>=2) {
    i=fib(n-1); j=fib(n-2);
    value[n] = i+j;
  }
  return value[n];
}
\end{verbatim}
Howver, this is inefficienty, since most intermediate values will be computed
more than once. We solve this by keeping track of which results are known:
\begin{verbatim}
  ...
  done = new int[nmax+1];
  for (i=0; i<=nmax; i++)
    done[i] = 0;
  done[0] = 1;
  done[1] = 1;
  ...
int fib(int n) {
  int i, j;
  if (!done[n]) {
    i = fib(n-1); j = fib(n-2);
    value[n] = i+j; done[n] = 1;
  }
  return value[n];
}
\end{verbatim}
The OpenMP parallel solution calls for two different ideas. First of all,
we parallelize the recursion by using tasks (section~\ref{sec:omp:task}:
\begin{verbatim}
int fib(int n) {
  int i, j;
  if (n>=2) {
#pragma omp task shared(i) firstprivate(n)
    i=fib(n-1);
#pragma omp task shared(j) firstprivate(n)
    j=fib(n-2);
#pragma omp taskwait
    value[n] = i+j;
  }
  return value[n];
}
\end{verbatim}
This computes the right solution, but, as in the naive single-threaded solution,
it recomputes many of the intermediate values.

A naive addition of the \n{done} array leads to data races, and probably an
incorrect solution:
\begin{verbatim}
int fib(int n) {
  int i, j, result;
  if (!done[n]) {
#pragma omp task shared(i) firstprivate(n)
    i=fib(n-1);
#pragma omp task shared(i) firstprivate(n)
    j=fib(n-2);
#pragma omp taskwait
    value[n] = i+j;
    done[n] = 1;
  }
  return value[n];
}
\end{verbatim}
For instance, there is no guarantee that the \n{done} array is updated
later than the \n{value} array, so a thread can think that \n{done[n-1]}
is true, but \n{value[n-1]} does not have the right value yet.

One solution to this problem is to use a lock, and make sure that,
for a given index~\n{n}, the values \n{done[n]} and \n{value[n]}
are never touched by more than one thread at a time:
\begin{verbatim}
int fib(int n)
{
  int i, j;
  omp_set_lock( &(dolock[n]) );
  if (!done[n]) {
#pragma omp task shared(i) firstprivate(n)
    i = fib(n-1);
#pragma omp task shared(j) firstprivate(n)
    j = fib(n-2);
#pragma omp taskwait
    value[n] = i+j;
    done[n] = 1;
  }
  omp_unset_lock( &(dolock[n]) );
  return value[n];
}
\end{verbatim}
This solution is correct, optimally efficient in the sense that it
does not recompute anything, and it uses tasks to obtain a parallel execution.

However, the efficiency of this solution is only up to a constant.
A~lock is still being set, even if a value is already computed and therefore
will only be read. This can be solved with a complicated use of critical sections,
but we will forego this.

\index{Fibonacci sequence|)}

\index{synchronization!in OpenMP|)}

