%%%%%%%%%%%%%%%%%%%%%%%%%%%%%%%%%%%%%%%%%%%%%%%%%%%%%%%%%%%%%%%%
%%%%%%%%%%%%%%%%%%%%%%%%%%%%%%%%%%%%%%%%%%%%%%%%%%%%%%%%%%%%%%%%
%%%%
%%%% This text file is part of the source of 
%%%% `Parallel Computing'
%%%% by Victor Eijkhout, copyright 2012/3
%%%%
%%%%%%%%%%%%%%%%%%%%%%%%%%%%%%%%%%%%%%%%%%%%%%%%%%%%%%%%%%%%%%%%
%%%%%%%%%%%%%%%%%%%%%%%%%%%%%%%%%%%%%%%%%%%%%%%%%%%%%%%%%%%%%%%%

\Level 0 {Discussion}

Hybrid computing decreases the number of messages.

On the other hand it makes the run more synchronous.

New version of Amdahl: sections that MPI-parallel but not OpenMP-parallel.

Allows overdecomposition on the node.

\Level 0 {Hybrid MPI-plus-threads execution}

In hybrid execution, the main question is whether all threads
are allowed to make MPI calls. To determine this,
replace the \n{MPI_Init} call by
\indexmpishow{MPI_Init_thread}:
\begin{verbatim}
int MPI_Init_thread
  ( int *argc, char ***argv, int required, int *provided )
\end{verbatim}
Here the \n{required} and \n{provided} parameters can take the following
values:
\begin{description}
\item[\texttt{MPI\_THREAD\_SINGLE}]\indexmpi{MPI_THREAD_SINGLE} Only a
  single thread will execute.
\item[\texttt{MPI\_THREAD\_FUNNELLED}]\indexmpi{MPI_THREAD_FUNNELLED}
  The program may use multiple threads, but only the mean thread will
  make MPI calls.
\item[\texttt{MPI\_THREAD\_SERIAL}]\indexmpi{MPI_THREAD_SERIAL} The
  program may use multiple threads, all of which may make MPI calls,
  but there will never be simultaneous MPI calls in more than one
  thread.
\item[\texttt{MPI\_THREAD\_MULTIPLE}]\indexmpi{MPI_THREAD_MULTIPLE}Multiple
  threads may MPI calls, without restrictions.
\end{description}

The \emph{mpirun}\index{mpirun!and environment variables}
program usually propagates \indexterm{environment variables},
so the value of \indextermtt{OMP_NUM_THREADS} when you call \n{mpirun}
will be seen by each MPI process.

\begin{itemize}
\item It is possible to use blocking sends in threads, and let the
  threads block. This does away with the need for polling.
\item You can not send to a thread number.
\end{itemize}

\begin{exercise}
Consider the 2D heat equation and explore the mix of MPI/OpenMP
parallelism:
\begin{itemize}
\item Give each node one MPI process that is fully multi-threaded.
\item Give each core an MPI process and don't use multi-threading.
\end{itemize}
Discuss theoretically why the former can give higher performance.
Implement both schemes as special cases of the general hybrid case,
and run tests to find the optimal mix.
\end{exercise}
