% -*- latex -*-
%%%%%%%%%%%%%%%%%%%%%%%%%%%%%%%%%%%%%%%%%%%%%%%%%%%%%%%%%%%%%%%%
%%%%%%%%%%%%%%%%%%%%%%%%%%%%%%%%%%%%%%%%%%%%%%%%%%%%%%%%%%%%%%%%
%%%%
%%%% This text file is part of the source of 
%%%% `Parallel Programming in MPI and OpenMP'
%%%% by Victor Eijkhout, copyright 2012-7
%%%%
%%%% mpi-data.tex : discussion of MPI datatypes
%%%%
%%%%%%%%%%%%%%%%%%%%%%%%%%%%%%%%%%%%%%%%%%%%%%%%%%%%%%%%%%%%%%%%
%%%%%%%%%%%%%%%%%%%%%%%%%%%%%%%%%%%%%%%%%%%%%%%%%%%%%%%%%%%%%%%%

\index{datatype|(}

In the examples you have seen so far, every time data was sent,
it was as a contiguous buffer with elements of a single type.
In practice you may want to send heterogeneous data, or
non-contiguous data.
\begin{itemize}
\item Communicating the real parts of an array of complex numbers
  means specifying every other number.
\item Communicating a C~structure of Fortran type with more than one
  type of element is not equivalent to sending an array of elements of
  a single type.
\end{itemize}
The datatypes you have dealt with so far are known as
\indextermsub{elementary}{datatypes}; irregular objects
are known as \indextermsub{derived}{datatypes}.

%% \Level 0 {Elementary data types}
\input chapters/mpi-elementary

%% \Level 0 {Derived datatypes}
\input chapters/mpi-derived

%% \Level 0 {More about data}
\input chapters/mpi-moredata

\index{datatype|)}

