% -*- latex -*-
%%%%%%%%%%%%%%%%%%%%%%%%%%%%%%%%%%%%%%%%%%%%%%%%%%%%%%%%%%%%%%%%
%%%%%%%%%%%%%%%%%%%%%%%%%%%%%%%%%%%%%%%%%%%%%%%%%%%%%%%%%%%%%%%%
%%%%
%%%% This text file is part of the source of 
%%%% `Parallel Programming in MPI and OpenMP'
%%%% by Victor Eijkhout, copyright 2012-6
%%%%
%%%% mpi-commbasic.tex : communicator basics
%%%%
%%%%%%%%%%%%%%%%%%%%%%%%%%%%%%%%%%%%%%%%%%%%%%%%%%%%%%%%%%%%%%%%
%%%%%%%%%%%%%%%%%%%%%%%%%%%%%%%%%%%%%%%%%%%%%%%%%%%%%%%%%%%%%%%%

\Level 0 {Duplicating communicators}
\label{sec:comm-dup}

With \indexmpishow{MPI_Comm_dup} you can make an exact duplicate of a communicator.
This may seem pointless, but it is actually very useful for the design of
software libraries. Image that you have a code
\begin{verbatim}
MPI_Isend(...); MPI_Irecv(...);
// library call
MPI_Waitall(...);
\end{verbatim}
and suppose that the library has receive calls. Now it is possible that the 
receive in the library inadvertently
catches the message that was sent in the outer environment.

First of all, here is code where the library stores the communicator
of the calling program:
%
\verbatimsnippet{wrongcatchlib}

To prevent this confusion, the library should duplicate the outer communicator,
and send all messages with respect to its duplicate. Now messages from the user
code can never reach the library software, since they are on different communicators.

\mpiRoutineRef{MPI_Comm_dup}

\verbatimsnippet{rightcatchlib}
\verbatimsnippet{catchlibp}
