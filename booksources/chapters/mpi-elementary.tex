% -*- latex -*-
%%%%%%%%%%%%%%%%%%%%%%%%%%%%%%%%%%%%%%%%%%%%%%%%%%%%%%%%%%%%%%%%
%%%%%%%%%%%%%%%%%%%%%%%%%%%%%%%%%%%%%%%%%%%%%%%%%%%%%%%%%%%%%%%%
%%%%
%%%% This text file is part of the source of 
%%%% `Parallel Programming in MPI and OpenMP'
%%%% by Victor Eijkhout, copyright 2012-6
%%%%
%%%% mpi-elementary.tex : elementary datatypes
%%%%
%%%%%%%%%%%%%%%%%%%%%%%%%%%%%%%%%%%%%%%%%%%%%%%%%%%%%%%%%%%%%%%%
%%%%%%%%%%%%%%%%%%%%%%%%%%%%%%%%%%%%%%%%%%%%%%%%%%%%%%%%%%%%%%%%

\Level 0 {Elementary data types}
\label{sec:elementary}
\index{datatype!elementary|(}

MPI has a number of elementary data types, corresponding to the 
simple data types of programming languages.
The names are made to resemble the types of C and~Fortran, 
for instance \n{MPI_FLOAT} and \n{MPI_DOUBLE} versus
\n{MPI_REAL} and \n{MPI_DOUBLE_PRECISION}.

MPI calls accept arrays of elements:
\begin{verbatim}
double x[20];
MPI_Send( x,20,MPI_DOUBLE, ..... )
\end{verbatim}
so for a single element you need to take its address:
\begin{verbatim}
double x;
MPI_Send( &x,1,MPI_DOUBLE, ..... )
\end{verbatim}

\Level 1 {C/C++}
\index{datatype!elementary!in C}

\begin{tabular}{|ll|}
  \hline
\n{MPI_CHAR}&only for text data, do not use for small integers\\
\n{MPI_UNSIGNED_CHAR}&\\
\n{MPI_SIGNED_CHAR}&\\
\n{MPI_SHORT}&\\
\n{MPI_UNSIGNED_SHORT}&\\
\n{MPI_INT}&\\
\n{MPI_UNSIGNED}&\\
\n{MPI_LONG}&\\
\n{MPI_UNSIGNED_LONG}&\\
\n{MPI_FLOAT}&\\
\n{MPI_DOUBLE}&\\
\n{MPI_LONG_DOUBLE}&\\
  \hline
\end{tabular}

There is some, but not complete, support for \indexterm{C99} types.

\Level 1 {Fortran}
\index{datatype!elementary!in Fortran}

\begin{tabular}{|ll|}
  \hline
\n{MPI_CHARACTER}&Character(Len=1)\\
\n{MPI_LOGICAL}&\\
\n{MPI_INTEGER}&\\
\n{MPI_REAL}&\\
\n{MPI_DOUBLE_PRECISION}&\\
\n{MPI_COMPLEX}&\\
\n{MPI_DOUBLE_COMPLEX}&Complex(Kind=Kind(0.d0))\\
  \hline
\end{tabular}

Addresses have type \indexmpishow{MPI_Aint} or \n{INTEGER
(KIND=MPI_ADDRESS_KIND)} in Fortran. The start of the address range is
given in \indexmpishow{MPI_BOTTOM}.

\Level 1 {Python}
\index{datatype!elementary!in Python}

\begin{tabular}{|ll|}
  \hline
  mpi4py type&NumPy type\\
  \hline
  \n{MPI.INT}&\n{np.intc}\\
  \n{MPI.LONG}&\n{np.int}\\
  \n{MPI.FLOAT}&\n{np.float32}\\
  \n{MPI.DOUBLE}&\n{np.float64}\\
  \hline
\end{tabular}
\index{datatype!elementary|)}
