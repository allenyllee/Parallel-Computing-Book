% -*- latex -*-
%%%%%%%%%%%%%%%%%%%%%%%%%%%%%%%%%%%%%%%%%%%%%%%%%%%%%%%%%%%%%%%%
%%%%%%%%%%%%%%%%%%%%%%%%%%%%%%%%%%%%%%%%%%%%%%%%%%%%%%%%%%%%%%%%
%%%%
%%%% This text file is part of the source of 
%%%% `Parallel Programming in MPI and OpenMP'
%%%% by Victor Eijkhout, copyright 2012-9
%%%%
%%%% mpi-elementary.tex : elementary datatypes
%%%%
%%%%%%%%%%%%%%%%%%%%%%%%%%%%%%%%%%%%%%%%%%%%%%%%%%%%%%%%%%%%%%%%
%%%%%%%%%%%%%%%%%%%%%%%%%%%%%%%%%%%%%%%%%%%%%%%%%%%%%%%%%%%%%%%%

\Level 0 {Elementary data types}
\label{sec:elementary}
\index{datatype!elementary|(}

MPI has a number of elementary data types, corresponding to the 
simple data types of programming languages.
The names are made to resemble the types of C and~Fortran, 
for instance \indexmpishow{MPI_FLOAT} and \indexmpishow{MPI_DOUBLE} versus
\indexmpishow{MPI_REAL} and \indexmpishow{MPI_DOUBLE_PRECISION}.

MPI calls accept arrays of elements:
\begin{lstlisting}
double x[20];
MPI_Send( x,20,MPI_DOUBLE, ..... )
\end{lstlisting}
so for a single element you need to take its address:
\begin{lstlisting}
double x;
MPI_Send( &x,1,MPI_DOUBLE, ..... )
\end{lstlisting}

\Level 1 {C/C++}
\index{datatype!elementary!in C}

\begin{tabular}{|ll|}
  \hline
% char
\indexmpishow{MPI_CHAR}&only for text data, do not use for small integers\\
\indexmpishow{MPI_UNSIGNED_CHAR}&\\
\indexmpishow{MPI_SIGNED_CHAR}&\\
% int
\indexmpishow{MPI_SHORT}&\\
\indexmpishow{MPI_UNSIGNED_SHORT}&\\
\indexmpishow{MPI_INT}&\\
\indexmpishow{MPI_UNSIGNED}&\\
\indexmpishow{MPI_LONG}&\\
\indexmpishow{MPI_UNSIGNED_LONG}&\\
\indexmpishow{MPI_LONG_LONG_INT}&\\
% real
\indexmpishow{MPI_FLOAT}&\\
\indexmpishow{MPI_DOUBLE}&\\
\indexmpishow{MPI_LONG_DOUBLE}&\\
% other
\indexmpishow{MPI_BYTE}&\\
\indexmpishow{MPI_PACKED}&\\
  \hline
\end{tabular}

There is some, but not complete, support for \indexterm{C99} types.

\Level 1 {Fortran}
\index{datatype!elementary!in Fortran}
\lstset{style=reviewcode,language=Fortran} %pyskip

\begin{tabular}{|ll|}
  \hline
\indexmpishow{MPI_CHARACTER}&Character(Len=1)\\
% int
\indexmpishow{MPI_INTEGER}&\\
\indexmpishow{MPI_INTEGER1}&\\
\indexmpishow{MPI_INTEGER2}&\\
\indexmpishow{MPI_INTEGER4}&\\
\indexmpishow{MPI_INTEGER8}&\\
\indexmpishow{MPI_INTEGER16}&\\
% real
\indexmpishow{MPI_REAL}&\\
\indexmpishow{MPI_DOUBLE_PRECISION}&\\
\indexmpishow{MPI_REAL2}&\\
\indexmpishow{MPI_REAL4}&\\
\indexmpishow{MPI_REAL8}&\\
% complex
\indexmpishow{MPI_COMPLEX}&\\
\indexmpishow{MPI_DOUBLE_COMPLEX}&Complex(Kind=Kind(0.d0))\\
% other
\indexmpishow{MPI_LOGICAL}&\\
\indexmpishow{MPI_PACKED}&\\
  \hline
\end{tabular}

Not all these types need be supported, for instance
\indexmpishow{MPI_INTEGER16} may not exist, in which case it will be
equivalent to \indexmpishow{MPI_DATATYPE_NULL}.

The default integer type \indexmpishow{MPI_INTEGER} is equivalent to
\lstinline{INTEGER(KIND=MPI_INTEGER_KIND)}.

Addresses have type \indexmpishow{MPI_Aint} or
\lstinline{INTEGER(KIND=MPI_ADDRESS_KIND)}
in Fortran. The start of the address range is
given in \indexmpishow{MPI_BOTTOM}.

There is also
\indexmpidef{MPI_COUNT_KIND},
\indexmpidef{MPI_OFFSET_KIND}.

\Level 2 {Fortran90 kind-defined types}
\label{sec:f90-types}

If your Fortran code uses \lstinline{KIND} to define scalar types with
specified precision, these do not in general correspond to any
predefined MPI datatypes. Hence the following routines exist to make
\emph{MPI equivalences of Fortran scalar types}%
\index{Fortran!MPI equivalences of scalar types}:
\indexmpiref{MPI_Type_create_f90_integer}
\indexmpiref{MPI_Type_create_f90_real}
\indexmpiref{MPI_Type_create_f90_complex}.

Examples:
\begin{lstlisting}
INTEGER ( KIND = SELECTED_INTEGER_KIND(15) ) , &
 DIMENSION(100) :: array INTEGER :: root , integertype , error 

CALL MPI_Type_create_f90_integer( 15 , integertype , error )
CALL MPI_Bcast ( array , 100 ,
 & integertype , root ,
 & MPI_COMM_WORLD , error )

REAL ( KIND = SELECTED_REAL_KIND(15 ,300) ) , &
 DIMENSION(100) :: array
CALL MPI_Type_create_f90_real( 15 , 300 , realtype , error )

COMPLEX ( KIND = SELECTED_REAL_KIND(15 ,300) ) , &
 DIMENSION(100) :: array 
CALL MPI_Type_create_f90_complex( 15 , 300 , complextype , error )
\end{lstlisting}
\lstset{style=reviewcode,language=C} %pyskip

\Level 1 {Python}
\index{datatype!elementary!in Python}

\begin{tabular}{|ll|}
  \hline
  mpi4py type&NumPy type\\
  \hline
  \n{MPI.INT}&\n{np.intc}\\
  \n{MPI.LONG}&\n{np.int}\\
  \n{MPI.FLOAT}&\n{np.float32}\\
  \n{MPI.DOUBLE}&\n{np.float64}\\
  \hline
\end{tabular}

\Level 1 {Byte addressing type}

So far we have mostly been taking about datatypes in the context of
sending them. The \indexmpidef{MPI_Aint} type is not so much for
sending, as it is for describing the size of objects, such as the size
of an \indexmpishow{MPI_Win} object; section~\ref{sec:windows}.

See also the \indexmpishow{MPI_Sizeof} and
\indexmpishow{MPI_Get_address} routines.

\Level 2 {Fortran}

The equivalent of
%\indexmpishowsub{MPI_Aint}{in Fortran}
\indexmpishowf{MPI_Aint} in Fortran
is an integer of kind \indexmpidef{MPI_ADDRESS_KIND}:
\lstset{style=reviewcode,language=Fortran} %pyskip
\begin{lstlisting}
integer(kind=MPI_ADDRESS_KIND) :: winsize
\end{lstlisting}
\lstset{style=reviewcode,language=C} %pyskip

Fortran lacks a \n{sizeof} operator to query the sizes of datatypes.
Since sometimes exact byte counts are necessary,
for instance in one-sided communication,
Fortran can use the \indexmpiref{MPI_Sizeof} routine.

Example usage in \indexmpishow{MPI_Win_create}:
\begin{lstlisting}
call MPI_Sizeof(windowdata,window_element_size,ierr)
window_size = window_element_size*500
call MPI_Win_create( windowdata,window_size,window_element_size,... );
\end{lstlisting}
\lstset{language=C} %pyskip

This routine is deprecated in \indexterm{MPI-4}: use of
\indextermtt{storage_size} and/or \indextermtt{c_sizeof} is recommended.

\Level 2 {Python}

The \indexmpishow{MPI_Win_create} routine needs a displacement in
bytes. Here is a good way for finding the size of \indexterm{numpy} datatypes:
\lstset{style=reviewcode,language=Python} %pyskip
\begin{lstlisting}
numpy.dtype('i').itemsize
\end{lstlisting}
\lstset{language=C} %pyskip

\index{datatype!elementary|)}

