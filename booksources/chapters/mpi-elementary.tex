% -*- latex -*-
%%%%%%%%%%%%%%%%%%%%%%%%%%%%%%%%%%%%%%%%%%%%%%%%%%%%%%%%%%%%%%%%
%%%%%%%%%%%%%%%%%%%%%%%%%%%%%%%%%%%%%%%%%%%%%%%%%%%%%%%%%%%%%%%%
%%%%
%%%% This text file is part of the source of 
%%%% `Parallel Programming in MPI and OpenMP'
%%%% by Victor Eijkhout, copyright 2012-9
%%%%
%%%% mpi-elementary.tex : elementary datatypes
%%%%
%%%%%%%%%%%%%%%%%%%%%%%%%%%%%%%%%%%%%%%%%%%%%%%%%%%%%%%%%%%%%%%%
%%%%%%%%%%%%%%%%%%%%%%%%%%%%%%%%%%%%%%%%%%%%%%%%%%%%%%%%%%%%%%%%

\Level 0 {Elementary data types}
\label{sec:elementary}
\index{datatype!elementary|(}

MPI has a number of elementary data types, corresponding to the 
simple data types of programming languages.
The names are made to resemble the types of C and~Fortran, 
for instance \indexmpishow{MPI_FLOAT} and \indexmpishow{MPI_DOUBLE} versus
\indexmpishow{MPI_REAL} and \indexmpishow{MPI_DOUBLE_PRECISION}.

MPI calls accept arrays of elements:
\begin{lstlisting}
double x[20];
MPI_Send( x,20,MPI_DOUBLE, ..... )
\end{lstlisting}
so for a single element you need to take its address:
\begin{lstlisting}
double x;
MPI_Send( &x,1,MPI_DOUBLE, ..... )
\end{lstlisting}

\Level 1 {C/C++}
\index{datatype!elementary!in C}

\begin{tabular}{|ll|}
  \hline
\n{MPI_CHAR}&only for text data, do not use for small integers\\
\n{MPI_UNSIGNED_CHAR}&\\
\n{MPI_SIGNED_CHAR}&\\
\n{MPI_SHORT}&\\
\n{MPI_UNSIGNED_SHORT}&\\
\n{MPI_INT}&\\
\n{MPI_UNSIGNED}&\\
\n{MPI_LONG}&\\
\n{MPI_UNSIGNED_LONG}&\\
\n{MPI_FLOAT}&\\
\n{MPI_DOUBLE}&\\
\n{MPI_LONG_DOUBLE}&\\
  \hline
\end{tabular}

There is some, but not complete, support for \indexterm{C99} types.

\Level 1 {Fortran}
\index{datatype!elementary!in Fortran}

\begin{tabular}{|ll|}
  \hline
\n{MPI_CHARACTER}&Character(Len=1)\\
\n{MPI_LOGICAL}&\\
\n{MPI_INTEGER}&\\
\n{MPI_REAL}&\\
\n{MPI_DOUBLE_PRECISION}&\\
\n{MPI_COMPLEX}&\\
\n{MPI_DOUBLE_COMPLEX}&Complex(Kind=Kind(0.d0))\\
  \hline
\end{tabular}

Addresses have type \indexmpishow{MPI_Aint} or \n{INTEGER
(KIND=MPI_ADDRESS_KIND)} in Fortran. The start of the address range is
given in \indexmpishow{MPI_BOTTOM}.

\Level 2 {Fortran90 kind-defined types}
\label{sec:f90-types}

If your Fortran code uses \n{KIND} to define scalar types with
specified precision, these do not in general correspond to any
predefined MPI datatypes. Hence the following routines exist to make
\emph{MPI equivalences of Fortran scalar types}%
\index{Fortran!MPI equivalences of scalar types}:
\indexmpiref{MPI_Type_create_f90_integer}
\indexmpiref{MPI_Type_create_f90_real}
\indexmpiref{MPI_Type_create_f90_complex}.

Examples:
\lstset{style=reviewcode,language=Fortran} %pyskip
\begin{lstlisting}
INTEGER ( KIND = SELECTED_INTEGER_KIND(15) ) , &
 DIMENSION(100) :: array INTEGER :: root , integertype , error 

CALL MPI_Type_create_f90_integer( 15 , integertype , error )
CALL MPI_Bcast ( array , 100 ,
 & integertype , root ,
 & MPI_COMM_WORLD , error )

REAL ( KIND = SELECTED_REAL_KIND(15 ,300) ) , &
 DIMENSION(100) :: array
CALL MPI_Type_create_f90_real( 15 , 300 , realtype , error )

COMPLEX ( KIND = SELECTED_REAL_KIND(15 ,300) ) , &
 DIMENSION(100) :: array 
CALL MPI_Type_create_f90_complex( 15 , 300 , complextype , error )
\end{lstlisting}
\lstset{style=reviewcode,language=C} %pyskip

\Level 1 {Python}
\index{datatype!elementary!in Python}

\begin{tabular}{|ll|}
  \hline
  mpi4py type&NumPy type\\
  \hline
  \n{MPI.INT}&\n{np.intc}\\
  \n{MPI.LONG}&\n{np.int}\\
  \n{MPI.FLOAT}&\n{np.float32}\\
  \n{MPI.DOUBLE}&\n{np.float64}\\
  \hline
\end{tabular}

\Level 1 {Byte addressing type}

So far we have mostly been taking about datatypes in the context of
sending them. The \indexmpidef{MPI_Aint} type is not so much for
sending, as it is for describing the size of objects, such as the size
of an \indexmpishow{MPI_Win} object; section~\ref{sec:windows}.

See also the \indexmpishow{MPI_Sizeof} routine.

\Level 2 {Fortran}

The equivalent of \indexmpishowsub{MPI_Aint}{in Fortran} is
\lstset{style=reviewcode,language=Fortran} %pyskip
\begin{lstlisting}
integer(kind=MPI_ADDRESS_KIND) :: winsize
\end{lstlisting}
\lstset{style=reviewcode,language=C} %pyskip

\Level 2 {Python}

The \indexmpishow{MPI_Win_create} routine needs a displacement in
bytes. Here is a good way for finding the size of \indexterm{numpy} datatypes:
\lstset{style=reviewcode,language=Python} %pyskip
\begin{lstlisting}
numpy.dtype('i').itemsize
\end{lstlisting}
\lstset{language=C} %pyskip

\index{datatype!elementary|)}

