\Level 0 {Review questions}

If you answer that a statement is false, give a one-line explanation.
\begin{enumerate}

%% system
\item True or false: \n{mpicc} is a compiler.

\item True or false: \n{mpirun} can only be used for interactive parallel runs.

\item What is the function of a hostfile?

%% communicators
\item True or false: in each communicator, processes are numbered consecutively from zero.

%% point-to-point
\item Describe a deadlock scenario involving three processors.

\item True or false: a message sent with \n{MPI_Isend} from one processor can be
  received with an \n{MPI_Recv} call on another processor.

\item True or false: a message sent with \n{MPI_Send} from one processor can be
  received with an \n{MPI_Irecv} on another processor.

\item Why does the \n{MPI_Irecv} call not have an \n{MPI_Status} argument?

%% one-sided
\item What is the relation between the concepts of `origin', `target', `fence',
  and `window' in one-sided communication.

\item What are the three routines for one-sided data transfer?

%% collective
\item Give an example of a collective call with and without a root processor.

\item Given a distributed array, meaning that every processor has
\begin{verbatim}
double x[N]; // N can vary per processor
\end{verbatim}
give the approximate MPI-based code that computes the maximum value
in the array, and leaves the result on every processor.

%% datatypes
\item Give two examples of derived datatypes.

\item Give a practical example where the sender uses a different type to send
  than the receiver uses in the corresponding receive call. Name the types involved.
\begin{comment}

%% theory
\item Give a simple model for the time a send operation takes.

\item Give a simple model for the time a broadcast of a single scalar takes.


\item 
\end{comment}
\end{enumerate}

