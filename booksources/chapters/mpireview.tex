% -*- latex -*-
%%%%%%%%%%%%%%%%%%%%%%%%%%%%%%%%%%%%%%%%%%%%%%%%%%%%%%%%%%%%%%%%
%%%%%%%%%%%%%%%%%%%%%%%%%%%%%%%%%%%%%%%%%%%%%%%%%%%%%%%%%%%%%%%%
%%%%
%%%% This text file is part of the source of 
%%%% `Parallel Programming in MPI and OpenMP'
%%%% by Victor Eijkhout, copyright 2012-8
%%%%
%%%% mpireview.tex : MPI review questions
%%%%
%%%%%%%%%%%%%%%%%%%%%%%%%%%%%%%%%%%%%%%%%%%%%%%%%%%%%%%%%%%%%%%%
%%%%%%%%%%%%%%%%%%%%%%%%%%%%%%%%%%%%%%%%%%%%%%%%%%%%%%%%%%%%%%%%

For all true/false questions, if you answer that a statement is false, give a one-line explanation.

\Level 0 {Conceptual}

\begin{exercise}
  \label{ex:m1}
  True or false: \n{mpicc} is a compiler.
\end{exercise}

\begin{exercise}
  What is the function of a hostfile?
\end{exercise}

\begin{pcse}
\begin{exercise}
    An architecture is called `symmetric' or `uniform' if the
    relation between any pair of processes is essentially the same.
    In what way are MPI processes run on stampede symmetric; in what way
    not?
\end{exercise}
\end{pcse}

\Level 0 {Communicators}

\begin{enumerate}

\item True or false: in each communicator, processes are numbered consecutively from zero.
\item If a process is in two communicators, it has the same rank in
both.

\end{enumerate}

\Level 0 {Point-to-point}

\begin{enumerate}
\item Describe a deadlock scenario involving three processors.

\item True or false: a message sent with \lstinline{MPI_Isend} from one processor can be
  received with an \lstinline{MPI_Recv} call on another processor.

\item True or false: a message sent with \lstinline{MPI_Send} from one processor can be
  received with an \lstinline{MPI_Irecv} on another processor.

\item Why does the \lstinline{MPI_Irecv} call not have an \lstinline{MPI_Status} argument?

%% one-sided
\item What is the relation between the concepts of `origin', `target', `fence',
  and `window' in one-sided communication.

\item What are the three routines for one-sided data transfer?

\lstset{
  style=reviewcode,
  language=C,
}

\item In the following fragments % in figures~\ref{fig:qblockc},\ref{fig:qblockf},
  assume that all buffers have been
  allocated with sufficient size. For each fragment note whether it
  deadlocks or not. Discuss performance issues.

%  \begin{figure}[p]
    \lstset{language=C,basicstyle=\footnotesize\ttfamily}
    \lstinputlisting{qblock1}
    \lstinputlisting{qblock2}
    \lstinputlisting{qblock3}
    \lstinputlisting{qblock4}
    \lstinputlisting{qblock5}
  %%   \caption{Do these C codes deadlock?}
  %%   \ref{fig:qblockc}
  %% \end{figure}

    \pagebreak
    Fortran codes:
    
    \lstset{style=reviewcode,language=Fortran}

%  \begin{figure}[p]
  \lstset{language=Fortran,basicstyle=\footnotesize\ttfamily}
  \lstinputlisting{qblock1f}
  \lstinputlisting{qblock2f}
  \lstinputlisting{qblock3f}
  \lstinputlisting{qblock4f}
  \lstinputlisting{qblock5f}
  %%   \caption{Do these Fortran codes deadlock?}
  %%   \ref{fig:qblockf}
  %% \end{figure}


\end{enumerate}

\Level 0 {Collectives}

\begin{enumerate}

\item MPI collectives can be divided into (a) rooted vs rootless (b) using uniform
buffer lengths vs variable length buffers (c) blocking vs non-blocking. Give examples of
each type.  

\item True or false: an \lstinline{MPI_Scatter} call puts the same data on
  each process.

\item Given a distributed array, with every processor storing
\begin{lstlisting}
double x[N]; // N can vary per processor
\end{lstlisting}
give the approximate MPI-based code that computes the maximum value
in the array, and leaves the result on every processor.

\item With data as in the previous question, given the code for
normalizing the array.
\end{enumerate}

\Level 0 {Datatypes}

\begin{enumerate}
\item Give two examples of MPI derived datatypes. What parameters are used
to describe them?

\item Give a practical example where the sender uses a different type to send
  than the receiver uses in the corresponding receive call. Name the types involved.

\item Fortran only. True or false?
  \begin{enumerate}
  \item Array indices can be different between the send and receive buffer arrays.
  \item It is allowed to send an array section.
  \item You need to \lstinline{Reshapce} a multi-dimensional array
    to linear shape before you can send it.
  \item An allocatable array, when dimensioned and allocated, is
    treated by MPI as if it were a normal static array, when used as
    send buffer.
  \item An allocatable array is allocated if you use it as the receive
    buffer: it is filled with the incoming data.
  \end{enumerate}
\item Fortran only: how do you handle the case where you want to use
  an allocatable array as receive buffer, but it has not been
  allocated yet, and you do not know the size of the incoming data?

\end{enumerate}

\Level 0 {Theory}

\begin{enumerate}

\item Give a simple model for the time a send operation takes.

\item Give a simple model for the time a broadcast of a single scalar takes.


\end{enumerate}

