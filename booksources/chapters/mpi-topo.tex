% -*- latex -*-
%%%%%%%%%%%%%%%%%%%%%%%%%%%%%%%%%%%%%%%%%%%%%%%%%%%%%%%%%%%%%%%%
%%%%%%%%%%%%%%%%%%%%%%%%%%%%%%%%%%%%%%%%%%%%%%%%%%%%%%%%%%%%%%%%
%%%%
%%%% This text file is part of the source of 
%%%% `Parallel Programming in MPI and OpenMP'
%%%% by Victor Eijkhout, copyright 2012-6
%%%%
%%%% mpi-topo.tex : about communicator topologies
%%%%
%%%%%%%%%%%%%%%%%%%%%%%%%%%%%%%%%%%%%%%%%%%%%%%%%%%%%%%%%%%%%%%%
%%%%%%%%%%%%%%%%%%%%%%%%%%%%%%%%%%%%%%%%%%%%%%%%%%%%%%%%%%%%%%%%

\Level 0 {Process topologies}
\label{sec:topology}

In the communicators you have seen so far, processes are linearly ordered.
In some circumstances the problem you are coding has some structure,
and expressing the program
in terms of that structure would be convenient. For this purpose, MPI 
can define a virtual \indexterm{topology}. There are two types:
\begin{itemize}
\item regular, Cartesian, grids; and
\item general graphs.
\end{itemize}

\Level 1 {Cartesian grid topology}
\label{sec:cartesian}

A \indextermsub{Cartesian}{grid} is a structure, typically in 2~or~3 dimensions,
of points that have two neighbours in each of the dimensions.
Thus, if a Cartesian grid has sizes $K\times M\times N$, its
points have coordinates $(k,m,n)$ with $0\leq k<K$ et cetera.
Most points have six neighbours $(k\pm1,m,n)$, $(k,m\pm1,n)$, $(k,m,n\pm1)$;
the exception are the edge points. A~grid where edge processors
are connected through \indexterm{wraparound connections} is called
a \indextermsub{periodic}{grid}.

The most common use of Cartesian coordinates
is to find the rank of process by referring to it in grid terms.
For instance, one could ask `what are my neighbours offset by $(1,0,0)$, 
$(-1,0,0)$, $(0,1,0)$ et cetera'.

