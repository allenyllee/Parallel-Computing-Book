% -*- latex -*-
%%%%%%%%%%%%%%%%%%%%%%%%%%%%%%%%%%%%%%%%%%%%%%%%%%%%%%%%%%%%%%%%
%%%%%%%%%%%%%%%%%%%%%%%%%%%%%%%%%%%%%%%%%%%%%%%%%%%%%%%%%%%%%%%%
%%%%
%%%% This text file is part of the source of 
%%%% `Parallel Programming in MPI and OpenMP'
%%%% by Victor Eijkhout, copyright 2012-9
%%%%
%%%% petsc-design.tex : a tutorial section
%%%%
%%%%%%%%%%%%%%%%%%%%%%%%%%%%%%%%%%%%%%%%%%%%%%%%%%%%%%%%%%%%%%%%
%%%%%%%%%%%%%%%%%%%%%%%%%%%%%%%%%%%%%%%%%%%%%%%%%%%%%%%%%%%%%%%%

\Level 0 {What is PETSc and why?}

PETSc is a library with a great many uses, but for now let's say that
it's primarily a library for dealing with the sort of linear algebra
that comes from discretized \acp{PDE}. On a single processor, the
basics of such computations 
can be coded out by a grad student during a semester
course in nummerical analysis, but on large scale issues get much more
complicated.

PETSc's prime justification is then that it helps you realize
scientific computations at large scales, meaning large problem sizes
on large numbers of processors.

There are two points to emphasize here:
\begin{itemize}
\item Linear algebra with dense matrices is relatively simple to
  formulate. For sparse matrices the amount of logistics in dealing
  with nonzero patterns increases greatly. PETSc does most of that for
  you.
\item Linear algebra on a single processor, even a multicore one, is
  managable; distributed memory parallelism is much harder, and
  distributed memory sparse linear algebra operations are doubly
  so. Using PETSc will save you many, many, Many! hours of coding over
  developing everything yourself from scratch.
\end{itemize}

\Level 1 {What is in PETSc?}

The routines in PETSc (of which there are hundreds) can roughly be
divided in these classes:
\begin{itemize}
\item Basic linear algebra tools: dense and sparse matrices, both
  sequential and parallel, their construction and simple operations.
\item Solvers for linear systems, and to a lesser extent nonlinear
  systems; also time-stepping methods.
\item Profiling and tracing: after a successful run, timing for
  various routines can be given. In case of failure, there are
  traceback and memory tracing facilities.
\end{itemize}

\Level 1 {Design philosophy}

PETSc has an object-oriented design, even though it is written
in~C. There are classes of objects, such Mat for matrices and Vec for
Vectors, but there is also the KSP (for “Krylov SPace solver”) class
of linear system solvers, and PetscViewer for outputting matrices and
vectors to screen or file.

Part of the object-oriented design is the polymorphism of objects:
after you have created a Mat matrix as sparse or dense, all methods
such as MatMult (for the matrix-vector product) take the same
arguments: the matrix, and an input and output vector.

This design where the programmer manipulates a `handle' also means
that the internal of the object, the actual storage of the elements,
is hidden from the programmer. This hiding goes so far that even
filling in elements is not done directly but through function calls:
\begin{lstlisting}
VecSetValue(i,j,v,mode)
MatSetValue(i,j,v,mode)
MatSetValues(ni,is,nj,js,v,mode)
\end{lstlisting}

\Level 1 {Language support}

PETSc is implemented in C, so there is a natural interface
to~C. A~\emph{Fortran90}\index{Fortran90!PETSc interface}
interface exists. The \emph{Fortran77}\index{Fortran77!PETSc interface}
interface is only of
interest for historical reasons.

A \emph{python}\index{Python!PETSc interface} interface was written by
Lisandro Dalcin, and requires separate installation, based on already
defined \indextermtt{PETSC_DIR} and \indextermtt{PETSC_ARCH}
variables.  This can be downloaded at
\url{https://bitbucket.org/petsc/petsc4py/src/master/}, with
documentation at
\url{https://www.mcs.anl.gov/petsc/petsc4py-current/docs/}.

\Level 1 {Documentation}

PETSc comes with a manual in pdf form and web pages with the
documentation for every routine. The starting point is the web page
\url{https://www.mcs.anl.gov/petsc/documentation/index.html}.

There is also a mailing list with excellent support for questions and
bug reports.
\begin{taccnote}
  For questions specific to using PETSc on TACC resources, submit
  tickets to the \emph{TACC}\index{TACC!portal} or
  \indextermbus{XSEDE}{portal}.
\end{taccnote}

\Level 0 {Basics of running a PETSc program}

\Level 1 {Compilation}

A PETSc compilation needs a number of include and library paths,
probably too many to specify interactively. The easiest solution is to
create a makefile:
\begin{verbatim}
include ${PETSC_DIR}/variables
include ${PETSC_DIR}/rules
program : program.o
        ${CLINKER} -o $@ $^ ${PETSC_LIB}
\end{verbatim}
The two include lines provide the compilation rule and the library
variable. If you want to write your own compiler rule, use
\begin{verbatim}
include ${PETSC_DIR}/variables
%.o : %.c
        ${CC} -c $^ ${PETSC_CC_INCLUDES}
program : program.o
        ${CLINKER} -o $@ $^ ${PETSC_LIB}
\end{verbatim}

The build process assumes that variables \indextermtt{PETSC_DIR} and
\indextermtt{PETSC_ARCH} have been set. These depend on your local
installation. Usually there will be one installation with debug
settings and one with production settings. Develop your code with the
former: it will do memory and bound checking. Then recompile and run
your code with the optimized production installation.

\begin{taccnote}
  On TACC clusters, a petsc installation is loaded by commands such as
\begin{verbatim}
module load petsc/3.11
\end{verbatim}
Use \n{module avail petsc} to see what configurations exist. The basic
versions are
\begin{verbatim}
# development
module load petsc/3.11-debug
# production
module load petsc/3.11
\end{verbatim}
Other installations are real versus complex, or 64bit integers instead
of the default 32. The command 
\begin{verbatim}
module spider petsc
\end{verbatim}
tells you all the
available petsc versions. The listed modules have a naming convention
such as \n{petsc/3.11-i64debug} where the 3.11 is the PETSc release (minor
patches are not included in this version; TACC aims to install only
the latest patch, but generally several versions are available), and
\n{i64debug} describes the debug version of the installation with 64bit
integers.
\end{taccnote}

\Level 1 {Running}

PETSc programs use MPI for parallelism, so they are run with 
\begin{verbatim}
mpirun -np 5 -machinefile mf your_petsc_program option1 option2 option3
\end{verbatim}
\begin{taccnote}
  On TACC clusters, use \indextermtt{ibrun}.
\end{taccnote}

\Level 1 {Startup}
\label{sec:petscinit}

PETSc has an call that initializes both PETSc and MPI,
so normally you would replace \indexmpishow{MPI_Init} by \indexpetscdef{PetscInitialize}.
Unlike with MPI, you do not want to use a \n{NULL} value for the
\n{argc,argv} arguments, since PETSc makes extensive use of
commandline options; see section~\ref{sec:petsc-options}.

\begin{verbatim}
ierr = PetscInitialize(&Argc,&Args,PETSC_NULL,PETSC_NULL);
\end{verbatim}

\begin{pythonnote}
  The following works if you don't need commandline options.
\begin{verbatim}
from petsc4py import PETSc
\end{verbatim}
To pass commandline arguments to PETSc, do:
\begin{verbatim}
import sys
from petsc4py import init
init(sys.argv)
from petsc4py import PETSc
\end{verbatim}
\end{pythonnote}

After initialization, you can use \indexmpishow{MPI_COMM_WORLD} or
\indexpetscshow{PETSC_COMM_WORLD}:

\begin{lstlisting}
MPI_Comm comm = PETSC_COMM_WORLD;
MPI_Comm_rank(comm,&mytid);
MPI_Comm_size(comm,&ntids);
\end{lstlisting}

\begin{pythonnote}
\begin{verbatim}
comm = PETSc.COMM_WORLD
nprocs = comm.getSize(self) 
procno = comm.getRank(self)
\end{verbatim}
\end{pythonnote}

\Level 1 {Printing}

Printing screen output in parallel is tricky. If two processes execute
a print statement at more or less the same time there is no guarantee
as to in what order they may appear on screen. (Even attempts to have
them print one after the other may not result in the right ordering.)
Furthermore, lines from multi-line print actions on two processes may
wind up on the screen interleaved.

PETSc has two routines that fix this problem. First of all, often the
information printed is the same on all processes, so it is enough if
only one process, for instance process~0, prints it.
%
\petscRoutineRef{PetscPrintf}

If all processes need to print, there is a routine that forces the
output to appear in process order.
%
\petscRoutineRef{PetscSynchronizedPrintf}

To make sure that output is properly flushed from all system buffers
use a flush routine:
%
\petscRoutineRef{PetscSynchronizedFlush}
%
where for ordinary screen output you would use \n{stdout} for the file.

\begin{pythonnote}
  Since the print routines use the python \n{print} call, they
  automatically include the trailing newline. You don't have to
  specify it as in the C~calls.
\end{pythonnote}

\Level 1 {Commandline options}

PETSc has as large number of commandline options, most of which we
will discuss later. For now we only mention \n{-log_summary} which
will print out profile of the time taken in various routines.
For these options to be parsed, it is necessary to pass \n{argc,argv}
to the \indexpetscshow{PetscInitialize} call.

