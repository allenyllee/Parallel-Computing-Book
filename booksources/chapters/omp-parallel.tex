% -*- latex -*-
%%%%%%%%%%%%%%%%%%%%%%%%%%%%%%%%%%%%%%%%%%%%%%%%%%%%%%%%%%%%%%%%
%%%%%%%%%%%%%%%%%%%%%%%%%%%%%%%%%%%%%%%%%%%%%%%%%%%%%%%%%%%%%%%%
%%%%
%%%% This text file is part of the source of 
%%%% `Parallel Programming in MPI and OpenMP'
%%%% by Victor Eijkhout, copyright 2012-9
%%%%
%%%% omp-parallel.tex : about parallel regions, nesting and stuff.
%%%%
%%%%%%%%%%%%%%%%%%%%%%%%%%%%%%%%%%%%%%%%%%%%%%%%%%%%%%%%%%%%%%%%
%%%%%%%%%%%%%%%%%%%%%%%%%%%%%%%%%%%%%%%%%%%%%%%%%%%%%%%%%%%%%%%%

\label{sec:parallelregion}
\index{parallel region|(}

The simplest way to create parallelism in OpenMP is to use
the \indexpragma{parallel} pragma. A~block preceded by the \n{omp parallel} pragma
is called a \emph{parallel region}; it
is executed by a newly created team of threads. 
This is an instance of the \indexac{SPMD} model: all threads execute the same
segment of code.
\begin{lstlisting}
#pragma omp parallel
{
  // this is executed by a team of threads
}
\end{lstlisting}
It would be pointless to have the block be executed identically by
all threads. One way to get a meaningful parallel code is to use the function
\indexompshow{omp_get_thread_num}, to find out which thread you are,
and execute work that is individual to that thread.
There is also a function
\indexompshow{omp_get_num_threads} to find out the total number of threads.
Both these functions give a number relative to the current team;
recall from figure~\ref{fig:forkjoin} that new teams can be created recursively.

For instance, if you program computes
\begin{lstlisting}
result = f(x)+g(x)+h(x)
\end{lstlisting}
you could parallelize this as
\begin{lstlisting}
double result,fresult,gresult,hresult;
#pragma omp parallel
{ int num = omp_get_thread_num();
  if (num==0)      fresult = f(x);
  else if (num==1) gresult = g(x);
  else if (num==2) hresult = h(x);
}
result = fresult + gresult + hresult;
\end{lstlisting}


The first thing we want to do is create a team of threads. This
is done with a \indexterm{parallel region}.
Here is a very simple example:
\cverbatimsnippet{hello-omp}
or in Fortran
\fverbatimsnippet{hello-who-omp-f}
This code corresponds to the model we just discussed:
\begin{itemize}
\item Immediately preceding the parallel block, one thread will be
  executing the code. In the main program this is the \emph{initial
    thread}\index{threads!initial}.
\item At the start of the block, a new \indextermsub{team of}{threads} is
  created, and the thread that was active before the block
  becomes the \emph{master thread}\index{threads!master} of that team.
\item After the block only the master thread is active.
\item Inside the block there is team of threads: each thread in the
  team executes the body of the block, and it will have access to all
  variables of the surrounding environment.
  How many
  threads there are can be determined in a number of ways; we will get to that later.
\end{itemize}

\begin{exercise}
  Make a full program based on this fragment. Insert different print statements
  before, inside, and after the parallel region.
  Run this example. How many times is each print statement executed?
\end{exercise}

You see that the \n{parallel} directive
\begin{itemize}
\item Is preceded by a special marker: a \n{#pragma omp} for~C/C++,
  and the \n{!$OMP} \indexterm{sentinel} for Fortran;
\item Is followed by a single statement or a block in~C/C++,
  or followed by a block in Fortran which is delimited by an \n{!$omp end} directive.
\end{itemize}

Directives look like \indextermsub{cpp}{directives}, but they
are actually handled by the compiler, not the preprocessor.

\begin{exercise}
  Take the `hello world' program above, and modify it so that you get
  multiple messages to you screen, saying
\begin{verbatim}
  Hello from thread 0 out of 4!
  Hello from thread 1 out of 4!
\end{verbatim}
  and so on. (The messages may very well appear out of sequence.)


  What happens if you set your number of threads larger than the available
  cores on your computer?
\end{exercise}

\begin{exercise}
  What happens if you call \n{omp_get_thread_num} and \n{omp_get_num_threads}
  outside a parallel region?
\end{exercise}

\begin{lstlisting}
  omp_get_thread_limit
\end{lstlisting}

\indexompshowdef{OMP_WAIT_POLICY} values: \n{ACTIVE,PASSIVE}

\Level 0 {Nested parallelism}
\index{nested parallelism|(}

What happens if you call a function from inside a parallel region, and
that function itself contains a parallel region?
\begin{lstlisting}
int main() {
  ...
#pragma omp parallel
  {
  ...
  func(...)
  ...
  }
} // end of main
void func(...) {
#pragma omp parallel
  {
  ...
  }
}
\end{lstlisting}

By default, the nested parallel region will have only one thread. To
allow nested thread creation, set
\begin{verbatim}
OMP_NESTED=true
 or
omp_set_nested(1)
\end{verbatim}

\begin{exercise}
  Test nested parallelism by writing an OpenMP program as follows:
  \begin{enumerate}
  \item Write a subprogram that contains a parallel region.
  \item\label{ex:nest:sub} Write a main program with a parallel region; call the subprogram both inside and outside the parallel region.
    \item Insert print statements 
      \begin{enumerate}
      \item in the main program outside the parallel region,
      \item in the parallel region in the main program,
      \item\label{ex:nest:sub:sub} in the subprogram outside the parallel region,
      \item in the parallel region inside the subprogram.
      \end{enumerate}
  \end{enumerate}
  Run your program and count how many print statements of each type you get.
\end{exercise}

Writing subprograms that are called in a parallel region illustrates
the following point: directives are evaluation with respect to the
\emph{dynamic scope}\index{parallel region!dynamic scope} of the
parallel region, not just the lexical scope. In the following example:
\begin{lstlisting}
#pragma omp parallel
{
  f();
}
void f() {
#pragma omp for
  for ( .... ) {
    ...
  }
}
\end{lstlisting}
the body of the function~\n{f} falls in the dynamic scope of the
parallel region, so the for loop will be parallelized.

If the function may be called both from inside and outside parallel
regions, you can test which is the case with \indexompshow{omp_in_parallel}.

The amount of nested parallelism can be set:
\begin{verbatim}
OMP_NUM_THREADS=4,2
\end{verbatim}
means that initially a parallel region will have four threads, and
each thread can create two more threads.

\begin{verbatim}
OMP_MAX_ACTIVE_LEVELS=123

omp_set_max_active_levels( n )
n = omp_get_max_active_levels()

OMP_THREAD_LIMIT=123

n = omp_get_thread_limit()

omp_set_max_active_levels
omp_get_max_active_levels
omp_get_level
omp_get_active_level
omp_get_ancestor_thread_num

omp_get_team_size(level)
\end{verbatim}

\index{nested parallelism|)}
\index{parallel region|)}

