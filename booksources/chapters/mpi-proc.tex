% -*- latex -*-
%%%%%%%%%%%%%%%%%%%%%%%%%%%%%%%%%%%%%%%%%%%%%%%%%%%%%%%%%%%%%%%%
%%%%%%%%%%%%%%%%%%%%%%%%%%%%%%%%%%%%%%%%%%%%%%%%%%%%%%%%%%%%%%%%
%%%%
%%%% This text file is part of the source of 
%%%% `Parallel Programming in MPI and OpenMP'
%%%% by Victor Eijkhout, copyright 2012-8
%%%%
%%%% mpi-proc.tex : about point-to-point communication
%%%%
%%%%%%%%%%%%%%%%%%%%%%%%%%%%%%%%%%%%%%%%%%%%%%%%%%%%%%%%%%%%%%%%
%%%%%%%%%%%%%%%%%%%%%%%%%%%%%%%%%%%%%%%%%%%%%%%%%%%%%%%%%%%%%%%%

\Level 0 {Process management}
\label{sec:mpi-dynamic}

The first version of MPI did not contain any process management
routines, even though the earlier \indexterm{PVM} project did have
that functionality. Process management was later added with MPI-2.

Unlike what you might think, newly added processes do not become part
of \n{MPI_COMM_WORLD}; rather, they get their own communicator, and an
\indexterm{intercommunicator} is established between this new group
and the existing one. The first routine is
\indexmpishow{MPI_Comm_spawn}, which tries to fire up multiple copies
of a single named executable. You could imagine using this mechanism
to start the whole of your MPI code, but that is likely to be inefficient.

\mpiRoutineRef{MPI_Comm_spawn}

(If you're feeling sure of yourself, specify \indexmpishow{MPI_ERRCODES_IGNORE}.)

Here is an example of a work manager.
%
\cverbatimsnippet{spawnmanager}
%
\pverbatimsnippet{spawnmanagerp}
%
You could start up a single copy of this program with 
\begin{verbatim}
mpirun -np 1 spawn_manager
\end{verbatim}
but with a hostfile that has more than one host. In that case the \indexmpishow{MPI_UNIVERSE_SIZE}
will tell you to the total number of hosts available. If this option
is not supported, you can determine yourself how many processes you
want to spawn. If you exceed the hardware resources, your
multi-tasking operating system (which is some variant of Unix for
almost everyone) will use \indexterm{time-slicing}, but you will not
gain any performance.

The spawned program looks very much like a regular MPI program, with
its own initialization and finalize calls.

\cverbatimsnippet{spawnworker}
%
\pverbatimsnippet{spawnworkerp}

Spawned processes wind up with a value of \n{MPI_COMM_WORLD} of their
own, but managers and workers can find each other regardless.
The spawn routine returns the intercommunicator to the parent; the children
can find it through \indexmpishow{MPI_Comm_get_parent}. The number of
spawning processes can be found through
\indexmpishow{MPI_Comm_remote_size} on the parent communicator.

\mpiRoutineRef{MPI_Comm_remote_size}

\Level 1 {MPMD}

Instead of spawning a single executable, you can spawn multiple with
\indexmpishow{MPI_Comm_spawn_multiple}.

\Level 1 {Socket-style communications}

\indexmpishow{MPI_Comm_connect}
\indexmpishow{MPI_Comm_accept}

\indexmpishow{MPI_Open_port}
\indexmpishow{MPI_Close_port}
\indexmpishow{MPI_Publish_name}
\indexmpishow{MPI_Unpublish_name}
\indexmpishow{MPI_Comm_join}
\indexmpishow{MPI_Comm_disconnect}

