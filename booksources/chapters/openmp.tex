% -*- latex -*-
%%%%%%%%%%%%%%%%%%%%%%%%%%%%%%%%%%%%%%%%%%%%%%%%%%%%%%%%%%%%%%%%
%%%%%%%%%%%%%%%%%%%%%%%%%%%%%%%%%%%%%%%%%%%%%%%%%%%%%%%%%%%%%%%%
%%%%
%%%% This text file is part of the source of 
%%%% `Parallel Programming in MPI and OpenMP'
%%%% by Victor Eijkhout, copyright 2012-6
%%%%
%%%%%%%%%%%%%%%%%%%%%%%%%%%%%%%%%%%%%%%%%%%%%%%%%%%%%%%%%%%%%%%%
%%%%%%%%%%%%%%%%%%%%%%%%%%%%%%%%%%%%%%%%%%%%%%%%%%%%%%%%%%%%%%%%

%\Level 0 {Basics}

%\Level 0 {Loop parallelism}

%\Level 0 {Work sharing}

%\Level 0 {Controlling thread data}

%\Level 0 {Reductions}

%\Level 0 {Synchronization}

% \Level 0 {Tasks}

\Level 0 {Runtime functions and internal control variables}
\label{ref:omp-environ}
\index{OpenMP!environment variables|(textbf}
\index{OpenMP!library routines|(textbf}
\index{Internal Control Variable (ICV)|(textbf}

OpenMP has a number of settings that can be set through \emph{environment variables},
and both queried and set through \emph{library routines}. These settings are called
\emph{\acfp{ICV}}: an OpenMP implementation behaves as if there is an internal variable
storing this setting.

The runtime functions are:
\begin{itemize}
\item \indextermttdef{omp_set_num_threads}
\item \indextermttdef{omp_get_num_threads}
\item \indextermttdef{omp_get_max_threads}
\item \indextermttdef{omp_get_thread_num}
\item \indextermttdef{omp_get_num_procs}
\item \indextermttdef{omp_in_parallel}
\item \indextermttdef{omp_set_dynamic}
\item \indextermttdef{omp_get_dynamic}
\item \indextermttdef{omp_set_nested}
\item \indextermttdef{omp_get_nested}
\item \indextermttdef{omp_get_wtime}
\item \indextermttdef{omp_get_wtick}
\item \indextermttdef{omp_set_schedule}
\item \indextermttdef{omp_get_schedule}
\item \indextermttdef{omp_set_max_active_levels}
\item \indextermttdef{omp_get_max_active_levels}
\item \indextermttdef{omp_get_thread_limit}
\item \indextermttdef{omp_get_level}
\item \indextermttdef{omp_get_active_level}
\item \indextermttdef{omp_get_ancestor_thread_num}
\item \indextermttdef{omp_get_team_size}
%\item \indextermttdef{omp_}
\end{itemize}

Here are the OpenMP \emph{environment variables}:
\begin{itemize}
\item \indextermttdef{OMP_CANCELLATION} Set whether cancellation is activated
\item \indextermtt{OMP_DISPLAY_ENV} Show OpenMP version and environment variables
\item \indextermttdef{OMP_DEFAULT_DEVICE} Set the device used in target regions
\item \indextermttdef{OMP_DYNAMIC} Dynamic adjustment of threads
\item \indextermttdef{OMP_MAX_ACTIVE_LEVELS} Set the maximum number of nested parallel
  regions
\item \indextermttdef{OMP_MAX_TASK_PRIORITY} Set the maximum task priority value
\item \indextermttdef{OMP_NESTED} Nested parallel regions
\item \indextermttdef{OMP_NUM_THREADS} Specifies the number of threads to use
\item \indextermttdef{OMP_PROC_BIND} Whether theads may be moved between CPUs
\item \indextermttdef{OMP_PLACES} Specifies on which CPUs the theads should be placed
\item \indextermttdef{OMP_STACKSIZE} Set default thread stack size
\item \indextermttdef{OMP_SCHEDULE} How threads are scheduled
\item \indextermttdef{OMP_THREAD_LIMIT} Set the maximum number of threads
\item \indextermttdef{OMP_WAIT_POLICY} How waiting threads are handled
\end{itemize}

There are 4 \acp{ICV} that behave as if each thread has its own copy of them.
The default is implementation-defined unless otherwise noted.
\begin{itemize}
  \item It may be possible to adjust dynamically the number of threads
    for a parallel region. Variable: \indextermtt{OMP_DYNAMIC};
    routines: \indextermtt{omp_set_dynamic},
    \indextermtt{omp_get_dynamic}.
  \item If a code contains \indextermsub{nested}{parallel regions},
    the inner regions may create new teams, or they may be executed by
    the single thread that encounters them. Variable:
    \indextermtt{OMP_NESTED}; routines \indextermtt{omp_set_nested},
    \indextermtt{omp_get_nested}. Allowed values are \n{TRUE} and
    \n{FALSE}; the default is false.
  \item The number of threads used for an encountered parallel region
    can be controlled. Variable: \indextermtt{OMP_NUM_THREADS};
    routines \indextermtt{omp_set_num_threads},
    \indextermtt{omp_get_max_threads}.
  \item The schedule for a parallel loop can be set. Variable:
    \indextermtt{OMP_SCHEDULE}; routines
    \indextermtt{omp_set_schedule}, \indextermtt{omp_get_schedule}.
\end{itemize}

Non-obvious syntax:
\begin{verbatim}
export OMP_SCHEDULE="static,100"
\end{verbatim}

Other settings:
\begin{itemize}
\item\indextermtt{omp_get_num_threads}: query the number of threads
  active at the current place in the code; this can be lower than what
  was set with \n{omp_set_num_threads}. For a meaningful answer, this
  should be done in a parallel region.
\item\indextermtt{omp_get_thread_num}
\item\indextermtt{omp_in_parallel}: test if you are in a parallel
  region (see for instance section~\ref{sec:parallelregion}).
\item\indextermtt{omp_get_num_procs}: query the physical number of cores available.
\end{itemize}

Other environment variables:
\begin{itemize}
\item \indextermtt{OMP_STACKSIZE} controls the amount of space that is
  allocated as per-thread \indexterm{stack}; the space for private
  variables.
\item \indextermtt{OMP_WAIT_POLICY} determines the behaviour of
  threads that wait, for instance for \indexterm{critical section}:
  \begin{itemize}
  \item \n{ACTIVE} puts the thread in a \indexterm{spin-lock}, where
    it actively checks whether it can continue;
  \item \n{PASSIVE} puts the thread to sleep until the \ac{OS} wakes
    it up.
  \end{itemize}
  The `active' strategy uses CPU while the thread is waiting; on the
  other hand, activating it after the wait is instantaneous. With the
  `passive' strategy, the thread does not use any CPU while waiting,
  but activating it again is expensive. Thus, the passive strategy
  only makes sense if threads will be waiting for a (relatively) long
  time.
\item \indextermtt{OMP_PROC_BIND} with values \n{TRUE} and \n{FALSE}
  can bind threads to a processor. On the one hand, doing so can
  minimize data movement; on the other hand, it may increase load
  imbalance.
\end{itemize}

\index{OpenMP!environment variables|)}
\index{OpenMP!library routines|)}
\index{Internal Control Variable (ICV)|)}

\Level 0 {Version 4 functionality}

\Level 1 {SIMD}

You can declare a loop to be executable with
\indextermbus{vector}{instructions} with
%
\indexpragmadef{simd}

The \indexpragma{simd} pragma has the following clauses:
\begin{itemize}
\item \indexclause{safelen($n$)}: limits the number of iterations in a
  SIMD chunk. Presumably useful if you combine \n{parallel for simd}.
\item \indexclause{linear}: lists variables that have a linear
  relation to the iteration parameter.
\item \indexclause{aligned}: specifies alignment of variables.
\end{itemize}

If your SIMD loop includes a function call, you can declare that the
function can be turned into vector instructions with
%
\indexpragma{declare simd}

If a loop is both multi-threadable and vectorizable, you can combine
directives as \n{pragma omp parallel for simd}.

\Level 0 {Stuff}

\Level 1 {Timing}
\label{sec:omp-timing}

OpenMP has a wall clock timer routine \indexcommand{omp_get_wtime}
with resolution \indexcommand{omp_get_wtick}.

\begin{exercise}
  Use the timing routines to demonstrate speedup from using
  multiple threads.
  \begin{itemize}
  \item Write a code segment that takes a measurable amount of time, that is,
    it should take a multiple of the tick time.
  \item Write a parallel loop and measure the speedup. You can for instance do this
\begin{verbatim}
for (int use_threads=1; use_threads<=nthreads; use_threads++) {
#pragma omp parallel for num_threads(use_threads)
    for (int i=0; i<nthreads; i++) {
        .....
    }
    if (use_threads==1)
      time1 = tend-tstart;
    else // compute speedup
\end{verbatim}
\item In order to prevent the compiler from optimizing your loop away, let
  the body compute a result and use a reduction to preserve these results.
  \end{itemize}
\end{exercise}

\Level 1 {Dependency analysis}
\index{data dependencies|(}
\index{flow dependency|see{data dependencies}}
\index{anti dependency|see{data dependencies}}
\index{output dependency|see{data dependencies}}

If two statements refer to the same data item,
we say that there is a \emph{data dependency} between
the statements. Such dependencies limit the extent to which
the execution of the statements can be  rearranged.
The study of this topic probably started in the 1960s,
when processors could execute statements \emph{out of order}\index{out-of-order execution}
to increase throughput. The re-ordering of statements
was limited by the fact that the execution
had to obey the \indexterm{program order} semantics:
the result had to be as if the statements were executed
strictly in the order in which they appear in the program.

These issues of statement ordering, and therefore of
data dependencies, arise in OpenMP in two main ways:
\begin{enumerate}
\item When a loop is parallelized, the iterations are no longer
  executed in their program order, so we have to check for dependencies.
\item The introduction of tasks also means that parts of a program
  can be executed in a different order from in which they appear
  in a sequential execution.
\end{enumerate}

The easiest case of dependency analysis is that of
detecting that loop iterations can be executed independently.
Iterations are of course independent if a data item
is read in two different iterations, but if the same
item is read in one iteration and written in another,
or written in two different iterations,
we need to do further analysis.

Analysis of \emph{data dependencies} can be performed
by a compiler, but compilers take, of necessity,
a conservative approach. This means that iterations
may be independent, but can not be recognized as such by
a compiler. Therefore, OpenMP shifts this responsibility
to the programmer; see for instance section~\ref{sec:omp-ordered}.

The three types of dependencies are:
\begin{itemize}
\item flow dependencies, or `read-after-write';
\item anti dependencies, or `write-after-read'; and
\item output dependencies, or `write-after-write'.
\end{itemize}

\begin{verbatim}
for (i) {
  y[i] = t;
  x[i+1] = y[i+1];
  t = x[i];
}
\end{verbatim}

\Level 2 {Flow dependencies}

Flow dependencies, or read-afer-write,
are not a problem if the read and write occur in the same
loop iteration:
\begin{verbatim}
for (i=0; i<N; i++) {
  x[i] = .... ;
  .... = ... x[i] ... ;
}
\end{verbatim}
On the other hand, if the read happens in a later iteration,
there is no simple way to parallelize the loop:
\begin{verbatim}
for (i=0; i<N; i++) {
  .... = ... x[i] ... ;
  x[i+1] = .... ;
}
\end{verbatim}
This usually requires rewriting the code.

\Level 2 {Anti dependencies}

The simplest case of write-after-read is a reduction:
\begin{verbatim}
for (i=0; i<N; i++) {
  t = t + .....
}
\end{verbatim}
This can be dealt with by explicit declaring the loop to be a reduction,
or to use any of the other strategies in section~\ref{sec:reduction}.

If the read and write are on an array the situation is more complicated.
The iterations in this fragment
\begin{verbatim}
for (i=0; i<N; i++) {
  x[i] = ... x[i+1] ... ;
}
\end{verbatim}
can not be executed in arbitrary order as such. However, conceptually there
is no dependency. We can solve this by introducing a temporary array:
\begin{verbatim}
for (i=0; i<N; i++)
  xtmp[i] = x[i];
for (i=0; i<N; i++) {
  x[i] = ... xtmp[i+1] ... ;
}
\end{verbatim}
This is an example of a transformation that a compiler is unlikely
to perform, since it can greatly affect the memory demands of the program.
Thus, this is left to the programmer.

\Level 2 {Output dependencies}

The case of write-after-write does not occur by itself:
if a variable is written twice in sequence without an intervening
read, the first write can be removed without changing the meaning of the program.
Thus, this case reduces to a flow dependency.

Other output dependencies can easily be removed. In the following code, \n{t}~can be
declared private, thereby removing the dependency.
\begin{verbatim}
for (i=0; i<N; i++) {
  t = f(i)
  s += t*t;
}
\end{verbatim}
If the final value of \n{t} is wanted, the \indexpragma{lastprivate} can be used.

\index{data dependencies|)}

\Level 1 {Thread safety}
\index{thread-safe|(}

With OpenMP it is relatively easy to take existing code and make
it parallel by introducing parallel sections. If you're careful
to declare the appropriate variables shared and private,
this may work fine. However, your code may include
calls to library routines that include a \indexterm{race condition};
such code is said not to be \emph{thread-safe}.

For example a routine
\begin{verbatim}
static int isave;
int next_one() {
 int i = isave;
 isave += 1;
 return i;
}

...
for ( .... ) {
  int ivalue = next_one();
}
\end{verbatim}
has a clear race condition, as the iterations of the loop
may get different \n{next_one} values, as they are supposed to,
or not. This can be solved by using an \indexpragma{critical}
pragma for the \n{next_one} call; another solution 
is to use an \indexpragma{threadprivate} declaration for \n{isave}.
This is for instance the right solution if  the \n{next_one}
routine implements a \indexterm{random number generator}.

\index{thread-safe|)}

\Level 1 {The OpenMP memory model}
\label{sec:omp:flush}

\Level 2 {Thread synchronization}

Let's do a \indexterm{producer-consumer} model\footnote{This example
  is from Intel's excellent OMP course by Tim Mattson}.  This can be
implemented with sections, where one section, the producer, sets a
flag when data is available, and the other, the consumer, waits until
the flag is set.
\begin{verbatim}
#pragma omp parallel sections
{
  // the producer
  #pragma omp section
  {
    ... do some producing work ...
    flag = 1;
  }
  // the consumer
  #pragma omp section
  {
    while (flag==0) { }
    ... do some consuming work ...
  }
}
\end{verbatim}
One reason this doesn't work, is that the compiler will see that the flag is never used
in the producing section, and that is never changed in the consuming section, so
it may optimize these statements, to the point of optimizing them away.

The producer then needs to do:
\begin{verbatim}
... do some producing work ...
#pragma omp flush
#pragma atomic write
  flag = 1;
#pragma omp flush(flag)
\end{verbatim}
and the consumer does:
\begin{verbatim}
#pragma omp flush(flag)
while (flag==0) {
  #pragma omp flush(flag)
}
#pragma omp flush
\end{verbatim}
This code strictly speaking has a \indexterm{race condition} on the \n{flag} variable.
It is better to use an \indexpragma{atomic} pragma here: the producer has
\begin{verbatim}
#pragma atomic write
  flag = 1;
\end{verbatim}
and the consumer:
\begin{verbatim}
while (1) {
  #pragma omp flush(flag)
  #pragma omp atomic read
    flag_read = flag
  if (flag_read==1) break;
}
\end{verbatim}

\Level 2 {Data races}
\index{data race|see{race condition}}
\index{race condition!in OpenMP|(}

OpenMP, being based on shared memory, has a potential for \emph{race
  conditions}. These happen when two threads access the same data
item. The problem with race conditions is that programmer convenience
runs counter to efficient execution. For this reason, OpenMP simply
does not allow some things that would be desirable.

The basic rule about multiple-thread access of a single data item is:
\begin{quote}
  Any memory location that is \emph{written} by one thread, can not be
  \emph{read} by another thread in the same parallel region, if no
  synchronization is done.
\end{quote}
To start with that last clause: any workshare construct ends with an
\indextermsub{implicit}{barrier}, so data written before that barrier
can safely be read after it.

As an illustration of a possible problem:
\begin{verbatim}
c = d = 0;
#pragma omp sections
{
#pragma omp section
  { a = 1; c = b; }
#pragma omp section
  { b = 1; d = a; }
}
\end{verbatim}
Under any reasonable interpretation of parallel execution,
the possible values for \n{c,d} are $1,1$ $0,1$ or~$1,0$.
This is known as \indexterm{sequential consistency}:
the parallel outcome is consistent with a sequential execution that
interleaves the parallel computations, respecting their local statement orderings.
(See also~\HPSCref{sec:seq-consist}.)

However, without synchronization, threads are allowed to maintain a value for a
variable locally that is not the same as the stored value. In this
example, that means that the thread executing the first section need
not write its value of~\n{a} to memory, and likewise \n{b}~in the
second thread, so $0,0$~is in fact a possible outcome.

In order to resolve multiple accesses:
\begin{enumerate}
\item Thread one reads the variable.
\item Thread one flushes the variable.
\item Thread two flushes the variable.
\item Thread two reads the variable.
\end{enumerate}

% \url{https://software.intel.com/es-es/forums/intel-moderncode-for-parallel-architectures/topic/610017}

\index{race condition!in OpenMP|)}
  
\Level 1 {Accelerators}

In OpenMP 4.0 there is support for offloading work to an
\emph{accelerator}\index{OpenMP!accelerator support in}
or
\emph{co-processor}\index{OpenMP!co-processor support in}:
\begin{verbatim}
#pragma omp target [clauses]
\end{verbatim}
with clauses such as
\begin{itemize}
\item \n{data}: place data
\item \n{update}: make data consistent between host and device
\end{itemize}

\Level 1 {SIMD}

OpenMP 4.0 has a way of indicating that a loop should not
be arbitrarily divided over threads, but should be executed
over simd lanes:
\begin{verbatim}
#pragma omp simd [clauses]
\end{verbatim}

\Level 1 {Overhead costs}

Code parallelization ideally divides the running time of your program
by the number of parallel processing entities. In practice, the following
factors counteract this.

\Level 2 {Amdahl effects}

Any code will have parts that are not parallelizable. Amdahl's law
(see~\HPSCref{sec:amdahl}) quantizes the effect this has on parallel speedup.
In an OpenMP code, the sections that are executed by a single thread
will play the role of the sequential part.

\Level 2 {Thread overhead}

At the start of an OpenMP program, a pool of threads is created. This
incurs a one-time overhead that will probably be amortized over the
total runtime.

Work sharing constructs act as if they create a new team of threads every time.
In practice, the program probably keeps a pool of threads around that are dormant
in between parallel sections. This means that there is no thread creation overhead
associated with the start of a parallel section.

\Level 2 {Load balance}

On the other hand, at the end of a work sharing construct there is a barrier,
so an unbalanced load distribution will decrease the parallel efficiency.
If loop iterations are not uniform in their running time, it may pay off
to use dynamic rather than static scheduling.

On the other other hand, dynamic scheduling has overhead of its own,
since it involves the operating system.

\Level 2 {Synchronization}

Various synchronization constructs, such as critical sections, as well as
dynamic loop scheduling, are realized through \indexterm{operating system}
functions. These are often quite costly, taking many thousands of cycles.
Thus, the \indextermsub{cost of a }{critical sections} goes far beyond
the Amdahl cost of the loss of paralellism. Critical sections should be used only
if the parallel work far outweighs it.

\Level 0 {Performance}

%See the EPCC benchmark suite~\cite{epcc-ompbench}.

The performance of an OpenMP code can be influenced by the following\footnote
{This section is inspired by a presentation by Alexei Strelchenko.}::
\begin{itemize}
\item Amdahl effects
\item Communication
\item Data affinity
\item Load imbalance
\item Synchronization
\end{itemize}

Sequential code must clearly be kept to a minimum.

Cache coherence induces communication. Some of that is unavoidable,
but see the next point.

Data is cached, so to minimize communication
access to it should be as much as possible
on the same core. This is known as affinity.

Load imbalance can be counteracted by using
different loop schedules. The loop should be
on as high a level as possible.

Barriers are a form of synchronization.
They are expensive by themselves, and they
expose load imbalance. Implicit barriers happen
at the end of worksharing constructs; they
can be removed with \n{nowait}.




\endinput

\Level 0 {Idioms}

\Level 1 {While loops}

\verbatimsnippet{linkedlist}

