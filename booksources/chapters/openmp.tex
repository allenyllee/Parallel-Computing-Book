%%%%%%%%%%%%%%%%%%%%%%%%%%%%%%%%%%%%%%%%%%%%%%%%%%%%%%%%%%%%%%%%
%%%%%%%%%%%%%%%%%%%%%%%%%%%%%%%%%%%%%%%%%%%%%%%%%%%%%%%%%%%%%%%%
%%%%
%%%% This text file is part of the source of 
%%%% `Parallel Computing'
%%%% by Victor Eijkhout, copyright 2012/3
%%%%
%%%%%%%%%%%%%%%%%%%%%%%%%%%%%%%%%%%%%%%%%%%%%%%%%%%%%%%%%%%%%%%%
%%%%%%%%%%%%%%%%%%%%%%%%%%%%%%%%%%%%%%%%%%%%%%%%%%%%%%%%%%%%%%%%

\Level 0 {Basics}

\Level 1 {OpenMP code structure}
\commandref{omp-code}

\Level 1 {Stuff}

\Level 2 {Critical sections}

There are two pragmas for critical sections: \indexpragma{critical} and \indexpragma{atomic}.
The second one is more limited but has performance advantages.

A \n{critical} section works by acquiring a lock, which carries a substantial overhead.
Furthermore, if your code has multiple critical sections, they are all mutually exclusive:
if a thread is in one critical section, the other ones are all blocked.

On the other hand, the syntax for \n{atomic} sections is limited, but such sections
are not exclusive and they can be more efficient, since they assume that there is a hardware
mechanism for making them critical.

The problem with \n{critical} sections being mutually exclusive can be mitigated by naming them:
\begin{verbatim}
#pragma omp critical (optional_name_in_parens)
\end{verbatim}
