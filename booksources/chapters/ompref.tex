% -*- latex -*-
%%%%%%%%%%%%%%%%%%%%%%%%%%%%%%%%%%%%%%%%%%%%%%%%%%%%%%%%%%%%%%%%
%%%%%%%%%%%%%%%%%%%%%%%%%%%%%%%%%%%%%%%%%%%%%%%%%%%%%%%%%%%%%%%%
%%%%
%%%% This text file is part of the source of 
%%%% `Parallel Computing'
%%%% by Victor Eijkhout, copyright 2012/3
%%%%
%%%%%%%%%%%%%%%%%%%%%%%%%%%%%%%%%%%%%%%%%%%%%%%%%%%%%%%%%%%%%%%%
%%%%%%%%%%%%%%%%%%%%%%%%%%%%%%%%%%%%%%%%%%%%%%%%%%%%%%%%%%%%%%%%

This section gives reference information and illustrative examples
of the use of MPI. While the code snippets given here should be enough,
full programs can be found in the repository for this book
\url{https://bitbucket.org/VictorEijkhout/parallel-computing-book}.

\Level 0 {Basics}

\Level 1 {OpenMP setup}
\commandreflabel{omp-code}

If you use OMP commands in a program file, be sure to include
the proper header file \indexterm{omp.h}.
\begin{verbatim}
#include "omp.h" // for C
\end{verbatim}

For Fortran:
\begin{verbatim}
use omp_lib
\end{verbatim}
The \indexterm{structured block} that follows a pragma is basically a block 
that gets executed as a whole: you can not jump into or out of it.

\Level 0 {Thread stuff}

\Level 1 {Creating parallel threads}
\commandreflabel{omp-parallel}

\indexpragma{parallel}
\verbatimsnippet{hello-who-omp}

\indexpragma{parallel for}

\Level 0 {Controlling thread data}
\commandreflabel{sec:ompdata}

\indexpragma{shared}

\Level 1 {Private data}

Data that is declared private with the \indexpragma{private} directive is
put on a separate \indextermbus{stack}{per thread}. The OpenMP standard
does not dictate the size of these stacks, but beware of \indextermbus{stack}{overflow}.
A~typical default
is a few megabyte; you can control it with the environment variable
\indextermtt{OMP_STACKSIZE}

\indexpragma{firstprivate}
\indexpragma{lastprivate}
\indexpragma{copyin}

\Level 0 {Parallel regions}
\commandreflabel{parallelregion}

\indexpragma{parallel}

\Level 1 {Loop parallelism}
\commandreflabel{omp-for}

A loop prefixed with the \indexpragma{for} or \indexpragma{parallel for} pragma
(they are synonymous) is automatically parallelized.
\begin{itemize}
\item The loop can not contains \n{break}, \n{return}, \n{exit} statements, or
  \n{goto} to a label outside the loop.
\item The \n{continue} statement is allowed.
\item The index update has to be an increment (or decrement) by a fixed amount.
\end{itemize}

\Level 1 {Reductions}
\commandreflabel{reduction}

Arithmetic reductions: $+,*,-,\max,\min$

Logical operator reductions: \n{&,&&,|,||,^}

Fortran: 

\Level 0 {Stuff}

\Level 1 {Timing}

To do \indextermsub{OpenMP}{timing} you can use any system utility;
however there is a dedicated routine \indexcommand{omp_get_wtime}
that express the time since some starting point as a double:
\begin{verbatim}
double omp_get_wtime(void);
\end{verbatim}
To measure a time difference:
\begin{verbatim}
double tstart,tend,duration;
tstart = omp_get_wtime();
// do stuff
tend = omp_get_wtime();
duration = tend-tstart;
\end{verbatim}
The timer resolution is given by:
\begin{verbatim}
double omp_get_wtick(void);
\end{verbatim}
