% -*- latex -*-
%%%%%%%%%%%%%%%%%%%%%%%%%%%%%%%%%%%%%%%%%%%%%%%%%%%%%%%%%%%%%%%%
%%%%%%%%%%%%%%%%%%%%%%%%%%%%%%%%%%%%%%%%%%%%%%%%%%%%%%%%%%%%%%%%
%%%%
%%%% This text file is part of the source of 
%%%% `Parallel Programming in MPI and OpenMP'
%%%% by Victor Eijkhout, copyright 2012-6
%%%%
%%%% mpi-running.tex : about running MPI programs
%%%%
%%%%%%%%%%%%%%%%%%%%%%%%%%%%%%%%%%%%%%%%%%%%%%%%%%%%%%%%%%%%%%%%
%%%%%%%%%%%%%%%%%%%%%%%%%%%%%%%%%%%%%%%%%%%%%%%%%%%%%%%%%%%%%%%%

\Level 0 {Starting and running MPI processes}
\label{sec:mpi-init}

The \ac{SPMD} model may be initially confusing. Even though there is
only a single source, compiled into a single executable,
the parallel run comprises a number of independently started MPI
processes (see section~\ref{sec:mpiexec} for the mechanism).

The following exercises are designed to give you an intuition for this
one-source-many-processes setup. In the first exercise you will see
that the mechanism for starting MPI programs starts up independent
copies. There is nothing in the source that says `and now you become parallel'.

The following exercise shows you that 
\begin{exercise}
  \label{ex:hello1}
  Write a `hello world' program, without any MPI in it,
  and run it in parallel with \indextermtt{mpiexec} or your local equivalent. 

\begin{tacc}
    (On TACC machines such as stampede, use \indextermtt{ibrun}, no
    processor count.)
\end{tacc}

  Explain the output.
\end{exercise}

To get a useful MPI program you need at least the calls \indexmpishow{MPI_Init}
and \indexmpishow{MPI_Finalize} surrounding your code. See section~\ref{sec:mpi-init} for their syntax.
\begin{pythonnote}
  There are no initialize and finalize calls: the \n{import} statement
  performs the initialization.
\end{pythonnote}

This may look a bit like declaring `this is the parallel part of a
program', but that's not true: again, the whole code is executed
multiple times in parallel.

\begin{exercise}
  \label{ex:hello2}
  Add the commands \indexmpishow{MPI_Init} and \indexmpishow{MPI_Finalize}
  to your code. Put three different print statements in your code: one before the init,
  one between init and finalize, and one after the finalize. Again explain the output.
\end{exercise}

In the following exercise you will print out the hostname
of each MPI process; see section~\ref{ref:context} for the syntax.
\begin{exercise}
  \label{ex:procname}
  Now use the command \indexmpidef{MPI_Get_processor_name}
  in between the
  init and finalize statement, and print out on what processor your process runs.
  Confirm that you are able to run a program that uses two different nodes.

  (The character buffer needs to be allocated by your, with size at
  least \indexmpidef{MPI_MAX_PROCESSOR_NAME}.)
\begin{tacc}
TACC nodes have a hostname \n{cRRR-CNN}, where RRR is the rack number, C is the chassis
number in the rack, and NN is the node number within the chassis. Communication
is faster inside a rack than between racks!
\end{tacc}
\end{exercise}

\Level 1 {Headers}

If you use MPI commands in a program file, be sure to include
the proper header file, \indexterm{mpi.h} or \indexterm{mpif.h}.
\begin{verbatim}
#include "mpi.h" // for C
#include "mpif.h" ! for Fortran
\end{verbatim}
For \emph{Fortran90}\index{Fortran!Fortran90}, many MPI installations
also have an MPI module, so you can write
\begin{verbatim}
use mpi
\end{verbatim}
The internals of these files can be different between MPI
installations, so you can not compile one file against one \n{mpi.h}
file and another file, even with the same compiler on the same machine,
against a different MPI.

\Level 1 {Initialization / finalization}
\label{sec:mpi-init}

Every MPI program has to start with \indextermbus{MPI}{initialization}:
%
\mpiRoutineRef{MPI_Init}
%
where \indextermtt{argc} and \indextermtt{argv} are the arguments
of a C language main program:
\begin{verbatim}
int main(int argc,char **argv) {
    ....
    return 0;
}
\end{verbatim}
(It is allowed to pass \n{NULL} for these arguments.)

The commandline arguments \n{argc} and \n{argv} are only guaranteed to
be passed to process zero, so the best way to pass commandline information
is by a broadcast (section~\ref{sec:bcast}).

Note that the \n{MPI_Init} call is one of the few that differs between C and Fortran:
the C~routine takes the commandline arguments, which Fortran lacks.

If MPI is used in a library, MPI can have already been initialized in a main program.
For this reason, one can test where \n{MPI_Init} has been called with
%
\mpiRoutineRef{MPI_Initialized}
%

The regular way to conclude an MPI program is:
%
\mpiRoutineRef{MPI_Finalize}
%
but an abnormal end to a run can be forced by\indexmpi{MPI_Abort}
\begin{verbatim}
MPI_Abort(comm,value);
\end{verbatim}
This aborts execution on all processes associated with the communicator,
but many implementations simply abort all processes. The \n{value} parameter
is returned to the environment.

The corresponding Fortran calls are
\begin{verbatim}
call MPI_Init(ierr)
// your code
call MPI_Finalize(ierr)
\end{verbatim}

You can test whether \n{MPI_Finalize} has been called with
%
\mpiRoutineRef{MPI_Finalized}

\Level 1 {Information about the run}

Once MPI has been initialized, the \indexmpishow{MPI_INFO_ENV} object contains:
\begin{itemize}
\item \n{command}
  Name of program executed.
\item  \n{argv}
  Space separated arguments to command.
\item  \n{maxprocs}
  Maximum number of MPI processes to start.
\item   \n{soft}
  Allowed values for number of processors.
\item   \n{host}
  Hostname.
\item   \n{arch}
  Architecture name.
\item   \n{wdir}
  Working directory of the MPI process.
\item   \n{file}
  Value is the name of a file in which additional information is specified.
\item   \n{thread_level}
  Requested level of thread support, if requested before the program started execution.
\end{itemize}
Note that these are the requested values; the running program can for instance
have lower thread support.

\Level 1 {Commandline arguments}

The \indexmpishow{MPI_Init} routines takes a reference to \indextermtt{argc}
and \indextermtt{argv} for the following reason: the \n{MPI_Init} calls
filters out the arguments to \indexterm{mpirun} or \indexterm{mpiexec},
thereby lowering the value of \n{argc} and elimitating some of the \n{argv}
arguments.

On the other hand, the commandline arguments that are meant for \n{mpiexec}
wind up in the \indexmpishow{MPI_INFO_ENV} object as a set of key/value pairs.

%% Any \indexterm{commandline argument} to the program can only be
%% guaranteed to be passed correctly to process zero. Here is a fragment
%% of code that shows use of commandline arguments.  The program
%% \n{examples/mpi/c/init.c} takes a single integer commandline argument.
%% If the user forgets to specify an argument of specifies~\n{-h}, a
%% usage message is printed and the program aborts, otherwise the
%% parameter is broadcast to all processes.  
%% %
%% \verbatimsnippet{usage}
%% %
%% For Fortran:
%% %
%% \verbatimsnippet{usage-f}


