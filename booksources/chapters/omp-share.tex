% -*- latex -*-
%%%%%%%%%%%%%%%%%%%%%%%%%%%%%%%%%%%%%%%%%%%%%%%%%%%%%%%%%%%%%%%%
%%%%%%%%%%%%%%%%%%%%%%%%%%%%%%%%%%%%%%%%%%%%%%%%%%%%%%%%%%%%%%%%
%%%%
%%%% This text file is part of the source of 
%%%% `Parallel Programming in MPI and OpenMP'
%%%% by Victor Eijkhout, copyright 2012-6
%%%%
%%%% omp-share.tex : work sharing
%%%%
%%%%%%%%%%%%%%%%%%%%%%%%%%%%%%%%%%%%%%%%%%%%%%%%%%%%%%%%%%%%%%%%
%%%%%%%%%%%%%%%%%%%%%%%%%%%%%%%%%%%%%%%%%%%%%%%%%%%%%%%%%%%%%%%%

\label{sec:work-sharing}

The declaration of a \indexterm{parallel region} establishes a team of
threads. This offers the possibility of parallelism, but to actually
get meaningful parallel activity you need something more.
OpenMP uses the concept of a \indexterm{work sharing
construct}: a way of dividing parallelizable work over a team of threads.
The work sharing constructs are:
\begin{itemize}
\item \texttt{for} (for~C) or
  \texttt{do} (for Fortran). The threads divide up the loop iterations among
  themselves; see~\ref{sec:omp-for}.
\item \texttt{sections} The threads divide a fixed number of sections
  between themselves; see section~\ref{sec:omp-sections}.
\item \texttt{single} The section is executed by a single thread;
  section~\ref{sec:omp-single}.
\item \texttt{task} See section~\ref{sec:omp:task}.
\item \texttt{workshare} Can parallelize Fortran array syntax;
  section~\ref{sec:fortran-workshare}.
\end{itemize}

\Level 0 {Sections}
\label{sec:omp-sections}

A parallel loop is an example of independent work units that are numbered.
If you have a pre-determined number of independent work units, 
the \indexpragma{sections} is more appropriate. In a \n{sections} construct
can be any number of \indexpragma{section} constructs. These need to be
independent, and they can be execute by any available thread in the current team,
including having multiple sections done by the same thread.
\begin{lstlisting}
#pragma omp sections
{
#pragma omp section
  // one calculation
#pragma omp section
  // another calculation
}
\end{lstlisting}

This construct can be used to divide large blocks of independent work.
Suppose that in the following line, both \n{f(x)} and \n{g(x)}
are big calculations:
\begin{lstlisting}
  y = f(x) + g(x)
\end{lstlisting}
You could then write
\begin{lstlisting}
double y1,y2;
#pragma omp sections
{
#pragma omp section
  y1 = f(x)
#pragma omp section
  y2 = g(x)
}
y = y1+y2;
\end{lstlisting}
Instead of using two temporaries, you could also use a critical
section; see section~\ref{sec:critical}.  However, the best solution
is have a \n{reduction} clause on the \n{sections} directive:
\begin{lstlisting}
  y = f(x) + g(x)
\end{lstlisting}
You could then write
\begin{lstlisting}
y = 0;
#pragma omp sections reduction(+:y)
{
#pragma omp section
  y += f(x)
#pragma omp section
  y += g(x)
}
\end{lstlisting}

\Level 0 {Single/master}
\label{sec:omp-single}

The \indexpragmadef{single} and \indexpragmadef{master} pragma
limit the execution of a block to a single thread. 
This can for instance be used to print tracing information
or doing \emph{I/O}\index{I/O!in OpenMP} operations.
\begin{lstlisting}
#pragma omp parallel
{
#pragma omp single
  printf("We are starting this section!\n");
  // parallel stuff
}
\end{lstlisting}
Another use of \n{single} is to perform initializations
in a parallel region:
\begin{lstlisting}
int a;
#pragma omp parallel
{
  #pragma omp single
    a = f(); // some computation
  #pragma omp sections
    // various different computations using a
}
\end{lstlisting}

The point of the single directive in this last example is that the
computation needs to be done only once, because of the shared memory.
Since it's a work sharing construct there is an \emph{implicit
  barrier}\index{implicit barrier!after single directive} after it,
which guarantees that all threads have the correct value in their
local memory (see section~\ref{sec:omp:flush}.

\begin{exercise}
  \label{ex:omp-single-mpi}
  What is the difference between this approach and how the same
  computation would be parallelized in MPI?
\end{exercise}

The \indexpragma{master} directive, also enforces execution
on a single thread, specifically the master thread of the team,
but it does not have the synchronization through the implicit barrier.

\begin{exercise}
  Modify the above code to read:
\begin{lstlisting}
int a;
#pragma omp parallel
{
  #pragma omp master
    a = f(); // some computation
  #pragma omp sections
    // various different computations using a
}
\end{lstlisting}
  This code is no longer correct. Explain.
\end{exercise}

Above we motivated the \n{single} directive as a way of initializing
shared variables. It is also possible to use \n{single} to initialize
private variables. In that case you add the \indexclausedef{copyprivate}
clause. This is a good solution if setting the variable takes~I/O.

\begin{exercise}
  Give two other ways to initialize a private variable, with all
  threads receiving the same value. Can you give scenarios where each
  of the three strategies would be preferable?
\end{exercise}

\Level 0 {Fortran array syntax parallelization}
\label{sec:fortran-workshare}

The \n{parallel do} directive is used to parallelize loops,
and this applies to both C and Fortran. However, Fortran also
has implied loops in its \emph{array syntax}\index{Fortran!array syntax}.
To parallelize array syntax you can use the \indexpragma{workshare}
directive.

The \indexpragma{workshare} directive exists only in Fortran.
It can be used to parallelize
the implied loops in \emph{array syntax}\index{Fortran!array syntax},
as well as  \emph{forall}\index{Fortran!forall loops} loops.

