%%%%%%%%%%%%%%%%%%%%%%%%%%%%%%%%%%%%%%%%%%%%%%%%%%%%%%%%%%%%%%%%
%%%%%%%%%%%%%%%%%%%%%%%%%%%%%%%%%%%%%%%%%%%%%%%%%%%%%%%%%%%%%%%%
%%%%
%%%% This text file is part of the source of 
%%%% `Introduction to High-Performance Scientific Computing'
%%%% by Victor Eijkhout, copyright 2012
%%%%
%%%% This book is distributed under a Creative Commons Attribution 3.0
%%%% Unported (CC BY 3.0) license and made possible by funding from
%%%% The Saylor Foundation \url{http://www.saylor.org}.
%%%%
%%%%
%%%%%%%%%%%%%%%%%%%%%%%%%%%%%%%%%%%%%%%%%%%%%%%%%%%%%%%%%%%%%%%%
%%%%%%%%%%%%%%%%%%%%%%%%%%%%%%%%%%%%%%%%%%%%%%%%%%%%%%%%%%%%%%%%

\newcommand\furtherreading{\Level 0 {Further Reading}\label{sec:furtherreading-\chapshortname}}
\newcommand\heading[1]{\paragraph{\textbf{#1}}}

{\catcode`\^^I=13 \globaldefs=1
 \newcommand\listing[2]{\begingroup\small\par\vspace{1ex}
  \catcode`\^^I=13 \def^^I{\leavevmode\hspace{40pt}}
  \noindent\fbox{#1}
  \verbatiminput{#2}\endgroup}
 \newcommand\codelisting[1]{\begingroup\small\par\vspace{1ex}
  \catcode`\^^I=13 \def^^I{\leavevmode\hspace{40pt}}
  \noindent\fbox{#1}
  \verbatiminput{#1}\endgroup}
}
\newcommand\inv{^{-1}}\newcommand\invt{^{-t}}
\newcommand\setspan[1]{[\![#1]\!]}
\newcommand\fp[2]{#1\cdot10^{#2}}
\newcommand\fxp[2]{\langle #1,#2\rangle}
\def\n#{\bgroup \catcode`\_=12 \catcode`\>=12 \catcode`\<=12
  \catcode`\&=12 \catcode`\^=12 \catcode`\~=12 \def\\{\char`\\}\relax
  \tt \let\next=}

\newcommand\diag{\mathop{\mathrm {diag}}}
\newcommand\argmin{\mathop{\mathrm {argmin}}}
\newcommand\defined{
  \mathrel{\lower 5pt \hbox{${\equiv\atop\mathrm{\scriptstyle D}}$}}}

\newcommand\bbP{\mathbb{P}}
\newcommand\bbR{\mathbb{R}}

\newtheorem{remark}{Remark}
\newtheorem{definition}{Definition}
\newtheorem{theorem}{Theorem}
\newtheorem{lemma}{Lemma}
%% \newenvironment{highermath}
%%     {\bigskip\begin{quotation}\noindent\emph{MMM}}
%%     {\end{quotation}\bigskip\noindent\ignorespaces}

\usepackage{acronym}
\newwrite\acrowrite
\openout\acrowrite=acronyms.tex
\def\acroitem#1#2{\acrodef{#1}{#2}
    \write\acrowrite{\message{defining #1}\noexpand\acitem{#1}{#2}}
}
\acroitem{AVX}{Advanced Vector Extensions}
\acroitem{BSP}{Bulk Synchronous Parallel}
\acroitem{CAF}{Co-array Fortran}
\acroitem{CUDA}{Compute-Unified Device Architecture}
\acroitem{DAG}{Directed Acyclic Graph}
\acroitem{DSP}{Digital Signal Processing}
\acroitem{FPU}{Floating Point Unit}
\acroitem{FFT}{Fast Fourier Transform}
\acroitem{FSA}{Finite State Automaton}
\acroitem{GPU}{Graphics Processing Unit}
\acroitem{HPC}{High-Performance Computing}
\acroitem{HPF}{High Performance Fortran}
\acroitem{ICV}{Internal Control Variable}
\acroitem{MIC}{Many Integrated Cores}
\acroitem{MIMD}{Multiple Instruction Multiple Data}
\acroitem{MPI}{Message Passing Interface}
\acroitem{MTA}{Multi-Threaded Architecture}
\acroitem{NUMA}{Non-Uniform Memory Access}
\acroitem{OS}{Operating System}
\acroitem{PGAS}{Partitioned Global Address Space}
\acroitem{PDE}{Partial Diffential Equation}
\acroitem{PRAM}{Parallel Random Access Machine}
\acroitem{RDMA}{Remote Direct Memory Access}
\acroitem{RMA}{Remote Memory Access}
\acroitem{SAN}{Storage Area Network}
\acroitem{SaaS}{Software as-a Service}
\acroitem{SFC}{Space-Filling Curve}
\acroitem{SIMD}{Single Instruction Multiple Data}
\acroitem{SIMT}{Single Instruction Multiple Thread}
\acroitem{SM}{Streaming Multiprocessor}
\acroitem{SMP}{Symmetric Multi Processing}
\acroitem{SOR}{Successive Over-Relaxation}
\acroitem{SP}{Streaming Processor}
\acroitem{SPMD}{Single Program Multiple Data}
\acroitem{SPD}{symmetric positive definite}
\acroitem{SSE}{SIMD Streaming Extensions}
\acroitem{TLB}{Translation Look-aside Buffer}
\acroitem{UMA}{Uniform Memory Access}
\acroitem{UPC}{Unified Parallel C}
\acroitem{WAN}{Wide Area Network}
\acresetall
\closeout\acrowrite

\newcommand{\indexterm}[1]{\emph{#1}\index{#1}}
\newcommand{\indextermtt}[1]{\texttt{\emph{#1}}\index{#1|n}}
%% \newcommand{\indextermfunctionn}[1]{\texttt{\emph{#1}}\index{#1|n}}
%% \def\indextermfunction#{\bgroup\catcode`\_=12\relax
%%   \def\nnext{\expandafter\indextermfunctionn\expandafter{\next}\egroup}%
%%   \afterassignment\nnext \def\next}
\let\indextermfunction\indextermtt

\newcommand{\indextermp}[1]{\emph{#1s}\index{#1}}
\newcommand{\indextermsub}[2]{\emph{#1 #2}\index{#2!#1}}
\newcommand{\indextermsubp}[2]{\emph{#1 #2s}\index{#2!#1}}
\newcommand{\indextermbus}[2]{\emph{#1 #2}\index{#1!#2}}
\newcommand{\indextermstart}[1]{\emph{#1}\index{#1|(}}
\newcommand{\indextermend}[1]{\index{#1|)}}
\newcommand{\indexstart}[1]{\index{#1|(}}
\newcommand{\indexend}[1]{\index{#1|)}}
\makeatletter
\newcommand\indexac[1]{\emph{\ac{#1}}%
  %\tracingmacros=2 \tracingcommands=2
  \edef\tmp{\noexpand\index{%
    \expandafter\expandafter\expandafter
        \@secondoftwo\csname fn@#1\endcsname%
    @\acl{#1} (#1)}}\tmp}
\newcommand\indexacp[1]{\emph{\ac{#1}}%
  %\tracingmacros=2 \tracingcommands=2
  \edef\tmp{\noexpand\index{%
    \expandafter\expandafter\expandafter
        \@secondoftwo\csname fn@#1\endcsname%
    @\acl{#1} (#1)}}\tmp}
\newcommand\indexacf[1]{\emph{\acf{#1}}%
  \edef\tmp{\noexpand\index{%
    \expandafter\expandafter\expandafter
        \@secondoftwo\csname fn@#1\endcsname
    @\acl{#1} (#1)}}\tmp}
\newcommand\indexacstart[1]{%
  \edef\tmp{\noexpand\index{%
    \expandafter\expandafter\expandafter
        \@secondoftwo\csname fn@#1\endcsname
    @\acl{#1} (#1)|(}}\tmp}
\newcommand\indexacend[1]{%
  \edef\tmp{\noexpand\index{%
    \expandafter\expandafter\expandafter
        \@secondoftwo\csname fn@#1\endcsname
    @\acl{#1} (#1)|)}}\tmp}
\makeatother

\def\chaptertitle{\csname\chaptername title\endcsname}
\def\chaptershorttitle{\csname\chaptername shorttitle\endcsname}

