% -*- latex -*-
%%%%%%%%%%%%%%%%%%%%%%%%%%%%%%%%%%%%%%%%%%%%%%%%%%%%%%%%%%%%%%%%
%%%%
%%%% This TeX file is part of the tutorial
%%%% `Introduction to the PETSc library'
%%%% by Victor Eijkhout, eijkhout@tacc.utexas.edu
%%%%
%%%% copyright Victor Eijkhout 2012-9
%%%%
%%%%%%%%%%%%%%%%%%%%%%%%%%%%%%%%%%%%%%%%%%%%%%%%%%%%%%%%%%%%%%%%

\documentclass[11pt]{beamer}

\beamertemplatenavigationsymbolsempty
\usepackage{beamerthemeTACC}
\parskip=.5\baselineskip plus .5\baselineskip
\event{STC 335/394}
\def\hpcsemester{Fall 2019}
\def\hpcteachers{Victor Eijkhout}

\input slidemacs
\input listingmacs
\input coursemacs
\usepackage{graphicx,multicol,amsmath}

%%%%
%%%% In/exclude this that and the other
%%%%
% \usepackage{comment} already in coursemacs
\excludecomment{longversion}
\excludecomment{higher}
% not for edit
\excludecomment{details}
\includecomment{shortversion}
\begin{longversion}
\excludecomment{shortversion}
\includecomment{details}
\end{longversion}
%

\includecomment{compat} %% VLE what?!

\def\n{\bgroup\catcode`\_=12 \catcode`\&=12 \catcode`\$=12 
    \catcode`\<=12 \catcode`\>=12 \catcode`\~=12
    \tt \let\next=}
\def\inv{^{-1}}

%%%%
%%%% Layout stuff
%%%%
\usepackage{enumitem}
\setlist{nosep,label=$\bullet$}
\parskip=12pt plus 2pt

%%%%
%%%% And here we go....
%%%%
\begin{document}
\title{Introduction to the PETSc library}
\author{Victor Eijkhout\\
\tt\normalsize eijkhout@tacc.utexas.edu}
\date{}
\frame{\titlepage}

%% \frame{ \frametitle{Outline}
%% \small
%% \tableofcontents
%% }

\sectionframe{Introduction}
\input introduction

\input parallel

\sectionframe{Practical matters}
\begin{frame}[containsverbatim]{Running at TACC}
  Get the course files:
\begin{verbatim}
cp ~train00/petsc-course.tgz .
tar fxz petsc-course.tgz
cd petsc_course_dir
\end{verbatim}
\end{frame}

\begin{frame}[containsverbatim]{Running at TACC}
  First:
\begin{verbatim}
module load petsc/3.12-debug
# later: module load petsc/3.12
\end{verbatim}
  Compute node for 2 hours, on 2 nodes with 12 cores total:
\begin{verbatim}
idev -t 2:0:0 -N 2 -n 12
\end{verbatim}
If you do this during an official TACC training,
there will probably be a reservation in place.
\end{frame}

\begin{frame}[containsverbatim]{How to make exercises}
  \begin{itemize}
  \item Directory: \n{exercises-3.10-c} or \n{f} or (maybe) \n{p}
  \item Type \n{make exercisename} to compile it
  \item Python: setup once per session
\begin{verbatim}
module load python
\end{verbatim}
    No compilation needed. Run:\\ \n{ibrun python yourprogram}
  \end{itemize}
\end{frame}

\input starting
%\input spmd
\input vector
\input matrix
\input iterative
\input grid
\input is

\begin{longversion}
  \input nonlinear
  \input higher
\end{longversion}

\input profiling

\end{document}

