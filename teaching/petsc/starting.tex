% -*- latex
%%%%%%%%%%%%%%%%%%%%%%%%%%%%%%%%%%%%%%%%%%%%%%%%%%%%%%%%%%%%%%%%
%%%%
%%%% This TeX file is part of the tutorial
%%%% `Introduction to the PETSc library'
%%%% by Victor Eijkhout, eijkhout@tacc.utexas.edu
%%%%
%%%% copyright Victor Eijkhout 2012-2020
%%%%
%%%%%%%%%%%%%%%%%%%%%%%%%%%%%%%%%%%%%%%%%%%%%%%%%%%%%%%%%%%%%%%%

\sectionframe{Getting started}

\frame[containsverbatim]{\frametitle{Include files}
C:
\begin{verbatim}
#include "petsc.h"
int main(int argc,char **argv)
\end{verbatim}
Fortran:
\begin{verbatim}
program basic
#include <petsc/finclude/petsc.h>
  use petsc
  implicit none
\end{verbatim}
Python:
\begin{verbatim}
from petsc4py import PETSc
\end{verbatim}
}

\frame[containsverbatim]{\frametitle{Variable declarations, C}
\begin{verbatim}
  KSP            solver;
  Mat            A;
  Vec            x,y;
  PetscInt       n = 20;
  PetscScalar    v;
  PetscReal      nrm;
\end{verbatim}
Note Scalar vs Real
}
\frame[containsverbatim]{\frametitle{Variable declarations, F}
\begin{verbatim}
  KSP            :: solver
  Mat            :: A
  Vec            :: x,y
  PetscInt       :: j(3)
  PetscScalar    :: mv
  PetscReal      :: nrm
\end{verbatim}
Much like in C; uses cpp
}

\frame[containsverbatim]{\frametitle{Library setup, C}
\cverbatimsnippet{petscinit}
Can replace \verb+MPI_Init+

General: Every routine has an error return. Catch that value!
}

\frame[containsverbatim]{\frametitle{Library setup, F}
\begin{verbatim}
  call PetscInitialize(PETSC_NULL_CHARACTER,ierr)
      CHKERRQ(ier)
// all the petsc work
  call PetscFinalize(ierr); CHKERRQ(ier)
\end{verbatim}
Error code is now final parameter. This holds for every PETSc routine.

\begin{itemize}
\item \n{CHKERRA} in main program
\item \n{CKKERRQ} in subprograms
\end{itemize}
}

\begin{exerciseframe}[hello]
Look up the function \n{PetscPrintf} and print a message\\
`This program runs on 27 processors'\\
from process zero.

\begin{itemize}
\item Start with the template code \n{hello.c}/\n{hello.F}
\item (or see slide~\ref{sl:ranksize})
\item Compile with \n{make hello}
\item Part two: use \n{PetscSynchronizedPrintf}
\end{itemize}
\end{exerciseframe}

\petscroutineslide{PetscPrintf}

\begin{frame}[containsverbatim]{About routine prototypes: C/C++}
  \label{sec:protos}
Prototype:
\begin{verbatim}
PetscErrorCode VecCreate(MPI_Comm comm,Vec *v);
\end{verbatim}
Use:
\begin{verbatim}
PetscErrorCode ierr;
MPI_Comm comm = MPI_COMM_WORLD;
Vec v;
ierr = VecCreate( comm,&vec ); CHKERRQ(ierr).
\end{verbatim}
(always good idea to catch that error code)
\end{frame}

\lstset{language=Fortran}
\begin{frame}[containsverbatim]{About routine prototypes: Fortran}
Prototype
\begin{verbatim}
Subroutine VecCreate( comm,v,ierr )
Type(MPI_Comm) :: comm
Vec            :: v
PetscErrorCode :: ierr
\end{verbatim}
Use:
\begin{verbatim}
Type(MPI_Comm) :: comm = MPI_COMM_WORLD
Vec            :: v
PetscErrorCode :: ierr

call VecCreate(comm,v,ierr)
\end{verbatim}
\begin{itemize}
\item Final parameter always error parameter. Do not forget!
\item MPI types are of often \lstinline{Type(MPI_Comm)} and such,
\item PETSc datatypes are handled through the preprocessor.
\end{itemize}
\end{frame}

\lstset{language=Python}
\begin{frame}[containsverbatim]{About routine prototypes: Python}
Prototype:
\begin{verbatim}
# object method
PETSc.Mat.setSizes(self, size, bsize=None)
# class method
PETSc.Sys.Print(type cls, *args, **kwargs)
\end{verbatim}
Use:
\begin{verbatim}
from petsc4py import PETSc
PETSc.Sys.Print("detecting n option")
A = PETSc.Mat().create(comm=comm)
A.setSizes( ( (None,matrix_size), (None,matrix_size) ) )
\end{verbatim}
\end{frame}

\lstset{language=C}
\begin{details}
\frame[containsverbatim]{\frametitle{Note to self}
\begin{verbatim}
  PetscInitialize
    (&argc,&args,0,"Usage: prog -o1 v1 -o2 v2\n");
\end{verbatim}
run as
\begin{verbatim}
  ./program -help
\end{verbatim}
This displays the usage note, plus all available petsc options.

Not available in Fortran
}

\frame[containsverbatim]{\frametitle{Routine start/end, C}
Debugging support:
\begin{verbatim}
PetscFunctionBeginUser;
// all statements
PetscFunctionReturn(0);
\end{verbatim}
leads to informative tracebacks.

(Only in C, not in Fortran)
}

\begin{frame}[containsverbatim]{Program parameters, C}
(I'm leaving out the \n{CHKERRQ(ierr)} in the examples,\\
but do use this in actual code)
\begin{verbatim}
ierr = PetscOptionsGetInt
    (PETSC_NULL,PETSC_NULL,"-n",&n,&flag); CHKERRQ(ierr);
ierr = PetscPrintf
    (comm,"Input parameter: %d\n",n); CHKERRQ(ierr);
\end{verbatim}
Read commandline argument, print out from processor zero\\
flag can be \n{PETSC_NULL}
\end{frame}

\frame[containsverbatim]{\frametitle{Program parameters, F}
\begin{verbatim}
  character*80      msg
  call PetscOptionsGetInt(
    PETSC_NULL_OPTIONS, PETSC_NULL_CHARACTER, &
      "-n",n,PETSC_NULL_BOOL,ierr)
  write(msg,10) n
 10 format("Input parameter:",i5)
  call PetscPrintf(PETSC_COMM_WORLD,msg,ierr)
\end{verbatim}
\begin{compat}
\n{PETSC_NULL_BOOL} will be introduced after 3.10.2
\end{compat}
Less elegant than \n{PetscPrintf} in C

Note the \n{PETSC_NULL_XXX}: Fortran has strict type checking.
}
\end{details}

\endinput

\frame[containsverbatim]{\frametitle{, C}
\begin{verbatim}
\end{verbatim}
}
\frame[containsverbatim]{\frametitle{, F}
\begin{verbatim}
\end{verbatim}
}
