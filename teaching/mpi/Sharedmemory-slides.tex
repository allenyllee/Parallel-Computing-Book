% -*- latex -*-
%%%%%%%%%%%%%%%%%%%%%%%%%%%%%%%%%%%%%%%%%%%%%%%%%%%%%%%%%%%%%%%%
%%%%%%%%%%%%%%%%%%%%%%%%%%%%%%%%%%%%%%%%%%%%%%%%%%%%%%%%%%%%%%%%
%%%%
%%%% This text file is part of the lecture slides for
%%%% `Parallel Computing'
%%%% by Victor Eijkhout, copyright 2012-9
%%%%
%%%% Sharedmemory-slides.tex : slides about mpi shared memory
%%%%
%%%%%%%%%%%%%%%%%%%%%%%%%%%%%%%%%%%%%%%%%%%%%%%%%%%%%%%%%%%%%%%%
%%%%%%%%%%%%%%%%%%%%%%%%%%%%%%%%%%%%%%%%%%%%%%%%%%%%%%%%%%%%%%%%

\begin{frame}[containsverbatim]\frametitle{Why shared memory?}
  \begin{itemize}
  \item Shared memory: two processes can access each other's
    memory through \n{double*} (and such) pointers.
  \item Limitation: only window memory.
  \item Non-use case: remote update. This has all the problems of
    traditional shared memory.
  \item Good use case: every process needs access to large read-only
    dataset\\ Example: ray tracing.
  \end{itemize}
\end{frame}

\begin{frame}[containsverbatim]\frametitle{Shared memory treatment in MPI}
  \begin{itemize}
  \item MPI uses optimizations for shared memory: copy instead of
    socket call
  \item One-sided offers `fake shared memory': yes, can access another
    process' data, but only through function calls.
  \item MPI-3 shared memory gives you a pointer to another process'
    memory, \emph{if that process is on shared memory}.
  \end{itemize}
\end{frame}

\begin{frame}[containsverbatim]\frametitle{Shared memory per cluster node}
\includegraphics[scale=.15]{ranger-node}
\begin{itemize}
\item Cluster node has shared memory
\item Memory is attached to specific socket
\item beware \ac{NUMA} effects
\end{itemize}
\end{frame}

\begin{frame}[containsverbatim]\frametitle{Shared memory interface}
  Here is the high level overview; details next.
  \begin{itemize}
  \item Use \indexmpishow{MPI_Comm_split_type} to find processes on the same
    shared memory
  \item Use \indexmpishow{MPI_Win_allocate_shared} to create a window between
    processes on the same shared memory
  \item Use \indexmpishow{MPI_Win_shared_query} to get pointer to another
    process' window data.
  \item You can now use \n{memcpy} instead of \indexmpishow{MPI_Put}.
  \end{itemize}
\end{frame}

\begin{frame}[containsverbatim]\frametitle{Discover shared memory}
  \begin{itemize}
  \item
    \indexmpishow{MPI_Comm_split_type} splits into communicators of same type.
  \item Only supported type: \indexmpishow{MPI_COMM_TYPE_SHARED} splitting by
    shared memory.
  \end{itemize}

  \cverbatimsnippet{commsplittype}
\end{frame}

\begin{frame}[containsverbatim]\frametitle{Allocate shared window}
Use \indexmpishow{MPI_Win_allocate_shared} to create a window that can
be shared;
\begin{itemize}
\item Has to be on a communicator on shared memory
\item memory will be allocated contiguously\\
  convenient for address arithmetic,\\ not for NUMA:
  set \indextermtt{alloc_shared_noncontig} true in
  \indextermtt{MPI_Info} object
\end{itemize}
\cverbatimsnippet[code/mpi/shared.c]{mpisharedwindow}
\end{frame}

\begin{frame}[containsverbatim]\frametitle{Get pointer to other windows}
  Use \indexmpishow{MPI_Win_shared_query}:
  \cverbatimsnippet[code/mpi/shared.c]{mpisharedpointer}
\end{frame}

\begin{frame}[containsverbatim]\frametitle{Boring example: heat equation}
  \cverbatimsnippet{allocateshared3pt}
\end{frame}

\begin{frame}[containsverbatim]\frametitle{Exciting example: bulk data}
  \begin{itemize}
  \item Application: ray tracing:\\
    large read-only data strcture describing the scene
  \item traditional MPI would duplicate:\\
    excessive memory demands
  \item Better: allocate shared data on process~0 of the shared communicator
  \item every else points to this object.
  \end{itemize}
\end{frame}

\begin{exerciseframe}[shareddata]
  \input ex:shareddata  
\end{exerciseframe}

\endinput

\begin{frame}[containsverbatim]\frametitle{}
\begin{lstlisting}
  
\end{lstlisting}
\end{frame}

