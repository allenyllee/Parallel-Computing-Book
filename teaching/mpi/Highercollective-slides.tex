% -*- latex -*-
%%%%%%%%%%%%%%%%%%%%%%%%%%%%%%%%%%%%%%%%%%%%%%%%%%%%%%%%%%%%%%%%
%%%%%%%%%%%%%%%%%%%%%%%%%%%%%%%%%%%%%%%%%%%%%%%%%%%%%%%%%%%%%%%%
%%%%
%%%% This text file is part of the lecture slides for
%%%% `Parallel Computing'
%%%% by Victor Eijkhout, copyright 2018-2020
%%%%
%%%% Highercollective-slides.tex : more slides about collective operations
%%%%
%%%%%%%%%%%%%%%%%%%%%%%%%%%%%%%%%%%%%%%%%%%%%%%%%%%%%%%%%%%%%%%%
%%%%%%%%%%%%%%%%%%%%%%%%%%%%%%%%%%%%%%%%%%%%%%%%%%%%%%%%%%%%%%%%

\sectionframe{User-defined operators}

\begin{exerciseframe}[onenorm]
  \input ex:one-norm-op
\end{exerciseframe}


\sectionframe{Non-blocking collectives}

\begin{frame}[containsverbatim]\frametitle{Non-blocking collectives}
  \begin{itemize}
  \item Collectives are blocking.
  \item Compare blocking/non-blocking sends:\\
    \indexmpishow{MPI_Send} $\rightarrow$ \indexmpishow{MPI_Isend}
  \item Non-blocking collectives:\\
    \indexmpishow{MPI_Bcast} $\rightarrow$ \indexmpishow{MPI_Ibcast}
  \item Use for overlap communication/computation
  \item Imbalance resilience
  \item Allows pipelining
  \end{itemize}
\end{frame}

\begin{frame}[containsverbatim]\frametitle{Use of non-blocking collectives}
  \begin{itemize}
  \item Similar calls, but output a request object:
\begin{lstlisting}
MPI_Isomething( <usual arguments>, MPI_Request *req);
\end{lstlisting}
  \item Calls return immediately;\\
    the usual story about buffer reuse
  \item Requires \lstinline{MPI_Wait}\texttt{...} for completion.
  \item Multiple collectives can complete in any order
  \item No guaranteed progress.
  \end{itemize}
\end{frame}

\protoslide{MPI_Ibcast}

\begin{frame}[containsverbatim]\frametitle{Overlapping collectives}
  Independent collective and local operations:
\[ y \leftarrow Ax + (x^tx)y \]
\begin{lstlisting}
MPI_Iallreduce( .... x ..., &request);
// compute the matrix vector product
MPI_Wait(request);
// do the addition
\end{lstlisting}
\end{frame}

\begin{exerciseframe}[procgridnonblock]
  \hyperlink{exprocgrid}{\beamergotobutton{Earlier procgrid exercise}}

  \input ex:procgridnonblock
\end{exerciseframe}

\sectionframe{Non-blocking barrier}

\begin{frame}[containsverbatim]\frametitle{Just what is a barrier?}
  \begin{itemize}
  \item Barrier is not \emph{time} synchronization but \emph{state}
    synchronization.
  \item Test on non-blocking barrier: `has everyone reached some
    state'
  \end{itemize}
\end{frame}

\begin{frame}[containsverbatim]\frametitle{Use case: adaptive refinement}
  \begin{itemize}
  \item Some processes decide locally to alter their structure
  \item \ldots~need to communicate that to neighbours
  \item Problem: neighbours don't know whether to expect update calls,
    if at all.
  \item Solution: do update calls, if any, then post barrier.\\
    Everyone probe for updates, test for barrier.    
  \end{itemize}
\end{frame}

\begin{frame}\frametitle{Use case: distributed termination detection}
  \begin{itemize}
  \item Distributed termination detection (Matocha and Kamp, 1998):\\
    draw a global conclusion with local operations
  \item Everyone posts the barrier when done;
  \item keeps doing local computation while testing for the barrier to
    complete
  \end{itemize}
\end{frame}

\protoslide{MPI_Ibarrier}

\begin{frame}[containsverbatim]\frametitle{}
\cverbatimsnippet{ibarrierpoll}
\end{frame}

\begin{exerciseframe}[ibarrierupdate]
  \begin{itemize}
  \item Let each process send to a random number of randomly chosen
    neighbours. Use \indexmpishow{MPI_Isend}.
  \item Write the main loop with the \indexmpishow{MPI_Test} call.
  \item Insert an \indexmpishow{MPI_Iprobe} call and process incoming messages.
  \item Can you make sure that all sends are indeed processed?
  \end{itemize}
\end{exerciseframe}

\begin{frame}[containsverbatim]\frametitle{Problem with `progress'}
  \begin{itemize}
  \item Problem: \indexmpishow{MPI_Test} is local
  \item Something needs to force the barrier information to propagate
  \item Solution: force progress with \indexmpishow{MPI_Iprobe}
  \item Frowny face: barrier completion takes much longer than you'd expect.
  \end{itemize}
\end{frame}

\endinput

\begin{frame}[containsverbatim]\frametitle{}
\begin{lstlisting}
  
\end{lstlisting}
\end{frame}

