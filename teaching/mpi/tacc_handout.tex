\documentclass[11pt]{artikel3}

\usepackage[pdftex]{hyperref}
\usepackage{pslatex}

\begin{document}
\title{Guide to the MPI labs}
\author{}\date{}
\maketitle

\begin{enumerate}
\item Log in to the designated machine: stampede2:
\begin{verbatim}
ssh -X -l yourname stampede2.tacc.utexas.edu
\end{verbatim}
\item Copy the course material to your directory:
\begin{verbatim}
cp ~train00/mpi_foundations2019.tgz .
\end{verbatim}
and unpack:
\begin{verbatim}
tar fxz mpi_foundations2019.tgz
\end{verbatim}
\item Go into the course directory:
\begin{verbatim}
cd mpi_course_dir
\end{verbatim}
You will find there the slides of the course, the textbook the course
is based on, and exercise directories for the languages~C, C++,
Fortran, Fortran2008, and Python. Pick your favourite.
\item Start an interactive session, ideally on 2 nodes, for the
  duration of this course:
\begin{verbatim}
idev -N 2 -n 20 -t 3:0:0
\end{verbatim}
(That is: two nodes with 20 processes total, for 3 hours.)\\
For people attending in person there will be reserved nodes. You will
get a question if you want to use them.
\item The exercises will be named in the course. For instance, there
  will be a `hello' exercise. The setup for this exercise will be in
  the file \texttt{hello.c/cxx/F90/py}. To compile the exercise, do
\begin{verbatim}
make hello
\end{verbatim}
This will give you a program that you can run in parallel with \texttt{ibrun}:
\begin{verbatim}
ibrun hello
\end{verbatim}
\end{enumerate}

\end{document}
