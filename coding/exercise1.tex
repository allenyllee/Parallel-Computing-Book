\documentclass[11pt]{artikel3}

\usepackage{pslatex}
\def\n{\bgroup\catcode`\_=12 \tt \let\next=}

\begin{document}
\title{MPI Programming exercise: timing}
\author{}\date{}\maketitle

Start with the \n{time_avg} program, which computes the average time
taken for a computation. 

The make file has a handy feature: if you type \n{make} it will not
actually do anything, but it will tell you all its capabilities. (Take
a look at the makefile. How is this effect attained?)

Compile the program with \n{make time_avg} and run it in an \n{idev}
session: \n{ibrun time_avg}.

Now do the following experiments.

\section{Timer accuracy}

The first thing you have to wonder about when you're doing timings is:
how much does the timer disturb the measurements. 
Find a way to measure how much time the timer itself takes.

\section{Collecting results}

Instead of just computing the average time for a computation, 
gather all the timings on the root process and print them all.

\section{Trace output}

Do the following steps \emph{on a login node} (you may want to keep
two windows open):
\begin{enumerate}
\item \n{make total_clean} to remove all binaries
\item \n{module load tau}
\item Rebuild your program. Do you notice anything different about the
  output?
\item Now try to get an \n{idev} session on more than one node: \n{idev -N 2}.
\item Instead of running with \n{idev}, do \n{make idevrun}. (Inspect
  the makefile to see what magic is going on there!)
\item You'll see that a directory has been created. If you now type
  \n{make tauplots}, a file with extension~\n{slog2} is created.
\item Invoke the \n{jumpshot} program on that \n{slog2} file.
\item What does it all mean? Experiment with zooming in; see what the
  different colours mean.
\end{enumerate}

\section{Report!}

Write up a report of what you've done here. Ideally, your report
will have some code snippets and screen shots.
Note: every
question/suggestion in this assignment is of course an opportunity for
writing a paragraph or two. 

\end{document}
