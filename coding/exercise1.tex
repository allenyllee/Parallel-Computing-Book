\documentclass[11pt]{artikel3}

\usepackage{pslatex}
\def\n{\bgroup\catcode`\_=12 \tt \let\next=}

\begin{document}
\title{MPI Programming exercise: timing}
\author{}\date{}\maketitle

Start with the \n{time_avg} program, which computes the average time
taken for a computation. Compile the program with \n{make time_avg}
and run it in an \n{idev} session.

Now do the following experiments.

\section{Timer accuracy}

The first thing you have to wonder about when you're doing timings is:
how much does the timer disturb the measurements. 
Find a way to measure how much time the timer itself takes.

\section{Collecting results}

Instead of just computing the average time for a computation, 
gather all the timings on the root process and print them all.

\end{document}
